\documentclass[a4paper]{article}
\usepackage[left=2.5cm, right=2.5cm,top=2.5cm,bottom=2.5cm]{geometry}

\usepackage[komastyle]{scrpage2}
\pagestyle{scrheadings}
\setheadsepline{0.5pt}[\color{black}]
% \usepackage[a4paper,margin=0.5in]{geometry}
%\setkomafont{captionlabel}{\sffamily\bfseries}
%\setcapindent{0em}
%\usepackage[ngerman]{babel}
\usepackage[utf8]{inputenc}
\usepackage{latexsym,exscale,stmaryrd,amssymb,amsmath}
\usepackage[nointegrals]{wasysym}
\usepackage{eurosym}
\usepackage{color}
\usepackage{rotating}
\usepackage{graphicx}
\usepackage{wrapfig}
% \usepackage{subfigure}
\usepackage{sidecap}
\usepackage{float}
\usepackage{hyperref}
\usepackage{subfig}
\usepackage{fancyvrb}
% \usepackage{subcaption}

\title{Spezialisierungspraktikum: \\ Computation of the Axion Relic Density}



\author{Janik Riess }
\date{April 2019}

\begin{document}

\maketitle

\section{Introduction}
In this term paper I will discuss the
computation of the axion relic density produced by the misalignment mechanism. The basic framework is
described and analytic solution and approximation are computed.
Furthermore the concepts for the numerical computation for more complicated models are layedout but not executed. Those will form the basics for the following bachelor thesis.

The axion is a hypothetical particle that was introduced as a standard model
extention to solve the strong CP problem using the Peccei-Quinn mechanism and axion like particles (ALPs) are also
introduced in string theory. There are extremly light pseudo scalar bosons. ???????

The misalignment mechanism is a non-thermal production mechanism for axion dark matter.
Here the axion field is due to symmetry breaking during inflation driven by a potential
to a mean value of zero today. ??????
The axion field and therefore there density is descried by a classical field equation.
This is justified my the low mass of the axion compared to the ????????
and therefore in high occupation numbers leading to averaged collective behavior
eg. classical motion. ???????
By solving the equation of motion one can compute the density of the axion dark matter
today as a function of the axion mass and the initial condition (field value)
of the axion field and show that the axion model can contribute/produce the density parameter as it is measured for our universe today. ????????

\section{The Equation of Motion}
In this section the classical equation of motion for the axion field is derived from the action of a scalar field. Then it is simplified for the background field in a homogeneous universe. Also the initial conditions are discussed.

\subsection{The Action}
The action for a scalar field on a curved background is given after \cite[Chap. 4.1, Page 25]{MarshAxionCosmo}
using the signature $(-, +, +, +)$ for the metric as
\begin{align}
    \label{eq:action}
    S[\phi] = \int \mathrm{d}^4 \sqrt{-g} \left(- \frac{1}{2} (\partial \phi)^2 - V(\phi) \right)
\end{align}
The measure $\mathrm{d}x^4 \sqrt{-g}$ accounds for the effect of a curved background while
the lagrange density depscribs a massiv scala field without self interaction
and a driving potential $V(\phi)$.

\subsection{The Klein Gordon Equation on a curved background}
Varying the action from equation \ref{eq:action} by $\phi \rightarrow \phi + \delta \phi$ gives

\begin{align*}
    S[\phi + \delta \phi] &= \int \mathrm{d} x^4 \sqrt{-g} \left( - \frac{1}{2} (\partial (\phi + \delta \phi))^2 - V(\phi + \delta \phi)  \right) \\
    &= \int \mathrm{d}x^4 \sqrt{-g} \left( - \frac{1}{2} g^{\mu \nu} (\partial_\mu \delta \phi \partial_\nu \phi + \partial_\nu \delta \phi \partial_\mu \phi) - V'(\phi) \delta \phi \right) \\
    &= - \frac{1}{2} \left( \int \mathrm{d}x^4 \sqrt{-g} g^{\mu \nu} \partial_\mu \delta \phi \partial_\nu \phi + \int \mathrm{d}x^4 \sqrt{-g} g^{\mu \nu} \partial_\nu \delta \phi \partial_\mu \phi \right) - \int \mathrm{d} x^4 \sqrt{-g} V'(\phi) \delta \phi \\
    &= - \frac{1}{2} \left( - \int \mathrm{d}x^4 \delta \phi \partial_\mu \sqrt{-g} g^{\mu \nu} \partial_\nu \phi +
                            - \int \mathrm{d}x^4 \delta \phi \partial_\nu \sqrt{-g} g^{\mu \nu} \partial_\mu \phi \right) - \int \mathrm{d} x^4 \sqrt{-g} V'(\phi) \delta \phi \\
    &= \int \mathrm{d} x^4 \delta \phi \left( \partial_\mu \sqrt{-g} g^{\mu \nu} \partial_\nu \phi - \sqrt{-g} V'(\phi) \right)
\end{align*}
using

\begin{align*}
    V(\phi + \delta \phi) \approx V(\phi) + \frac{\partial V(\phi)}{\partial \phi} \delta \phi + O(\delta \phi^2)
\end{align*}
and
\begin{align*}
      (\partial (\phi + \delta \phi))^2 &= (\partial^\mu (\phi + \delta \phi)) (\partial_\mu (\phi + \delta \phi)) \\
                                        &= g^{\mu \nu} (\partial_\nu (\phi + \delta \phi)) (\partial_\mu (\phi + \delta \phi)) \\
                                        &= g^{\mu \nu} (\partial_\nu \phi \partial_\mu \phi + \partial_\nu \phi \partial_\mu \delta \phi +
                                                        \partial_\nu \delta \phi \partial_\mu \phi + \partial_\nu \delta \phi \partial_\mu \delta \phi) \\
                                        &= \mathrm{const} + g^{\mu \nu} (\partial_\nu \phi \partial_\mu \delta \phi + \partial_\nu \delta \phi \partial_\mu \phi) + O(\delta \phi ^2)
\end{align*}
as well was integration by parts in the forth equal sign and the symmetry of $g^{\mu \nu}$ and renaming indicies in the last
equal sign.
The term linear in $\delta \phi$ has to vanish so
\begin{align}
    \label{eq:klein_gordon}
    \Box \phi = \frac{1}{\sqrt{-g}} \partial_\mu \sqrt{-g} g^{\mu \nu} \partial_\nu \phi = \frac{\partial V(\phi)}{\partial \phi}
\end{align}
the Klein Gordon Equation holds for the axion field. \cite[Chap. 4.1, Page 26]{MarshAxionCosmo}


\subsection{Equation for the background field}
We are interessted in the evolution of the background field $\bar{\phi}$ with $\phi = \bar{\phi} + \delta \phi$ where
$|\bar{\phi}| \gg |\delta \phi|$ and $\nabla \bar{\phi} = 0$. We therefore only have to consider the time dependence in equation \ref{eq:klein_gordon}. From now on $\phi$ will label the background field.
For the background field all derivatives $\partial_i$ vanish, therefore we can drop them in equation \ref{eq:klein_gordon} from the summation and only keep the time derivative.
\begin{align*}
    \frac{\partial V}{\partial \phi} &= \Box \phi \\
                                     &= \frac{1}{\sqrt{-g}} \partial_t \sqrt{-g} g^{00} \partial_t \phi \\
                                     &= \frac{1}{\sqrt{-g}} \left( (\partial_t \sqrt{-g} g^{00}) (\partial_t \phi) + \sqrt{-g} g^{00} \partial^2_t \phi \right)
\end{align*}
We are interested in the evolutation of the axion field in a homogenious universe which has the FLRW metric \cite[Chap. 1.3.2., Page 20]{CosmologyBookMukhanov}
\begin{align}
    \label{eq:flrw}
    \mathrm{d}s^2 = -\mathrm{d}t^2 + a(t)^2 \left(\frac{\mathrm{d}r^2}{1 - kr^2} + r^2(\mathrm{d}\theta^2 + \sin^2 \theta \mathrm{d} \varphi^2)\right)
\end{align}
with
\begin{align*}
    g^{00} = -1
\end{align*}
\begin{align*}
    g = -1 \cdot \frac{a(t)^2}{1 - kr^2} \cdot a(t)^2 r^2 \cdot a(t)^2 r^2 \sin^2 \theta
      = - a(t)^6 f(\vec{x})
\end{align*}
where $f$ is some function of the spatial coordinates.
So:
\begin{align*}
    \frac{\partial V}{\partial \phi} &= - \frac{1}{\sqrt{a(t)^6 f(\vec{x})}} (\partial_t \sqrt{ a(t)^6 f(\vec{x}) }) \dot{\phi} - \ddot{\phi}
                                     = - \frac{\partial_t a(t)^3}{a(t)^3} \dot{\phi} - \ddot{\phi} \\
                                     &= - \frac{\dot{a}(t) \cdot 3 a(t)^2}{a(t)^3} \dot{\phi} - \ddot{\phi}
                                     = - 3 \frac{\dot{a}}{a} \dot{\phi} - \ddot{\phi}
                                     = - 3H \dot{\phi} - \ddot{\phi}
\end{align*}
This yields the equation of motion for the axion field \cite[Chap 4.2, Page 25]{MarshAxionCosmo}
\begin{align}
    \label{eq:eom}
    \ddot{\phi} + 3 H \dot{\phi} + V'(\phi) = 0
\end{align}

\subsection{Initial Conditions}
Since equation \ref{eq:eom} is a second order ODE, we need the initial values
for $\phi$ and $\dot{\phi}$.
Because $m_a \ll H(t_i)$ the friction term in equation dominates over the
driving force from the potential term at early times.
So we can assume $\dot{\phi_i} = 0$.
The initial condition for $\phi_i$ is left as an open parameter that can be tuned to
reach the correct density of DM for today \cite[Chap. 4.2, Page 26]{MarshAxionCosmo}
TODO: fluctuations ????????????

\subsection{The Potential}
The potential $V(\phi)$ is unknown at this point, but it has a minimum
at $\phi = 0$. We therefore can taylor expand V around $\phi = 0$
and get the potential
\begin{align}
    \label{eq:potential}
    V(\phi) = \frac{1}{2} m_a^2 \phi^2
\end{align}
where $m_a$ is the axion mass as the coupling constant.
There are two remarks to be made.
First there might be higher oder anharmonic effects.
There are studied in section \ref{sec:anharmonic_effects}.
Second the axion mass can depend on the temperature of
the universe like in the case of the QCD Axion.
This effect is covered in section \ref{sec:temerature_dependent_mass}.
The Axion mass is also left open as a parameter to tune like
the initial value of the field.

\section{Solutions for the Klein Gordon Equation}
In the following sections there are different
situations for the potential as well as the
evolution of the scale parameter $a(t)$ for the
Klein Gordon equation discussed.

\subsection{Constant Mass and $a \propto t^p$}
In the case for a constant axion mass and
a power law for the scale parameter $a(t) \sim t^p$ like for radiation $p = \frac{1}{2}$
or matter domination $p = \frac{3}{4}$
the equation \ref{eq:eom} is analytically solvable \cite[Chap. 4.2, Page 25]{MarshAxionCosmo}.
We plug in the Ansatz
\begin{align*}
    \phi(t) &= \kappa \cdot t^m \cdot f(t) \\
    \dot{\phi}(t) &= \kappa \left( m t^{m - 1} f + t^m \dot{f} \right) \\
    \ddot{\phi}(t) &= \kappa \left( m (m - 1) t^{m - 2} f + 2 m t^{m - 1} \dot{f} + t^m \ddot{f} \right)
\end{align*}
into equation \ref{eq:eom}, dividing out $\kappa$ and get
\begin{align*}
    &\Rightarrow m (m - 1) t^{m - 2} f + 2 m t^{m - 1} \dot{f} + t^m \ddot{f} + 3H \left( m t^{m - 1} f + t^m \dot{f} \right) + m_a^2 t^m f = 0 \\
    &\Rightarrow m (m - 1) t^{- 2} f + 2 m t^{- 1} \dot{f} + \ddot{f} + 3H \left( m t^{- 1} f + t^m \dot{f} \right) + m_a^2 f = 0 \\
    &\Rightarrow f \left( m (m - 1) t^{-2} + 3H m t^{-1} + m_a^2 \right) + \dot{f} \left( 2mt^{-1} + 3 H \right) + \ddot{f} = 0 \\
    &\Rightarrow f \left( m (m - 1) + 3H m t + t^2 m_a^2 \right) + \dot{f} \left( 2mt + 3 H t^2 \right) + t^2 \ddot{f} = 0 \\
    % &\Rightarrow f \left( m (m - 1) + 3pm + t^2 m_a^2 \right) + \dot{f} \left( 2mt + 3 t p \right) + t^2 \ddot{f} = 0 \\
    &\Rightarrow f \left( m (m - 1) + 3pm + t^2 m_a^2 \right) + t \dot{f} \left( 2m + 3 p \right) + t^2 \ddot{f} = 0
\end{align*}
if
\begin{align*}
    2m + 3p = 1 \Rightarrow m = - 3 / 2 p + 1 / 2
\end{align*}
then
\begin{align*}
    m (m - 1) + 3pm &= (-3/2p + 1/2) (-3/2p - 1/2) + 3p(-3/2p + 1/2) \\
                    &= - (3p)^2 / 4 + p/2 - (1/2)^2 = - (\frac{3p + 1}{2})^2 =: -n^2
\end{align*}
resulting in
\begin{align*}
    (t^2 m_a^2 - n^2) f + t \dot{f} + t^2 \ddot{f} = 0
\end{align*}
for the unknown function $f$.
This equation get be recognized as the Bessel Equation which solution is by definition
the Bessel functions.
So the final solution is
\begin{align}
    \label{eq:power_law_solution}
    \phi(t) = a^{-3/2} \left(\frac{t}{t_i}\right)^{1/2}\left(C_1 J_n(m_a t) + C_2 Y_n(m_a t)\right)
\end{align}
Where $J_n$ and $Y_n$ are the Bessel function of first and second kind with
\begin{align}
    n = (3p - 1) / 2.
\end{align}
The constants $C_1$ and $C_2$ are fixed by the initial conditions from section ?????.
Plugging in the solution yields
\begin{align*}
    C_1 &= \frac{B \phi_i}{BC - AD} \\
    C_2 &= \frac{A \phi_i}{BC - AD} \\
    A &= \alpha J_n(m_a t_i) + m_a J_n'(m_a t_i) \\
    B &= \alpha Y_n(m_a t_i) + m_a Y_n'(m_a t_i) \\
    C &= a_i^{-3/2} J_n(m_a t_i) \\
    D &= a_i^{-3/2} Y_n(m_a t_i) \\
    \alpha &= (-\frac{3}{2} p + \frac{1}{2}) t_i^{-1}
\end{align*}
The solution was plotted for realistic values in figure \ref{fig:rad_dom_ax_field} for the radiation dominated case.
\begin{figure}
    \centering
    \includegraphics{analytic_power_law_plot.pdf}
    \caption{Evolution of the axion field in a radiation dominated universe. This plot is a replication of figure 4 in \cite{MarshAxionCosmo}}
    \label{fig:rad_dom_ax_field}
\end{figure}

Furthermore we can use the WKB approximation to gain a better understanding of the formation of the relic density \cite[Chap 4.3.1, Page 28]{MarshAxionCosmo}.
We use the Ansatz
\begin{align*}
    \phi &= A c \\
    \dot{\phi} &= \dot{A} c - s m_a A \\
    \ddot{\phi} &= \ddot{A} c - 2 m_a \dot{A} s - m_a^2 c A
\end{align*}
where $A = A(t)$ is a slowly varieing function of $t$ with $H / m_a \sim \dot{A} / m_a \sim \epsilon \ll 1$ and $\ddot{A} \ll 1$
and $c = \cos(m_a t + \theta)$, $s = \sin(m_a t + \theta) $
and plug it into equation \ref{eq:eom}
resulting in
\begin{align*}
    &\ddot{A} c - 2 m_a \dot{A} s - m_a^2 c A + 3 H ( \dot{A} c - m_a s A) + m_a^2 c A = 0 \\
    &\Rightarrow \underbrace{\frac{c}{m_a^2} \ddot{A}}_{\approx 0} - 2 s \underbrace{\frac{\dot{A}}{m_a}}_{= \epsilon} + 3c \underbrace{\frac{H \dot{A}}{m_a^2}}_{= \epsilon^2} - 3 s A \underbrace{\frac{H}{m_a}}_{= \epsilon} = 0 \\
\end{align*}
This gives
\begin{align*}
    \frac{\dot{A}}{A} = - \frac{3}{2} H = - \frac{3}{2} \frac{\dot{a}}{a}
\end{align*}
in leading order.
Using the Ansatz $A = a^n$ we get
\begin{align*}
    \frac{n \dot{a} a^{n - 1}}{a^n} = - \frac{3}{2} \frac{\dot{a}}{a} \Rightarrow n = - \frac{3}{2}
\end{align*}
So we can see that the axion field scales at later times like $\sim a^{-3/2}$
and oscillats with a frequency of $\omega = 2m_a$. So
the expectation value of the equation of state $\approx 0$ and the axion denity goes by
\begin{align*}
    \rho_a = \frac{1}{2}m_a^2\phi^2 \sim \left(a^{-3/2}\right)^2 = a^{-3}
\end{align*}
in this periode. Therefore the axion fields behaves like ordinary matter at this point and
is just diluted by the expantion of the universe, resulting in \cite[Chap. 4.3.1, Page 28]{MarshAxionCosmo}
\begin{align}
    \label{eq:wkb_axion_denity}
    \rho_a(a) = \rho_a(a_\mathrm{osc}) \left( \frac{a_\mathrm{osc}}{a} \right)^3
\end{align}
We can use this to compute analytic formulas for the axion density parameter for some cases \cite[Chap 4.3.1, Page 28]{MarshAxionCosmo} using the formula
\begin{align}
    \label{eq:hubble_parameter_evo}
    H(t)^2 = H_0^2 \left(
        \Omega_\mathrm{rad} \left( \frac{a_i}{a} \right)^4 +
        \Omega_\mathrm{mat} \left( \frac{a_i}{a} \right)^3
    \right)
\end{align}
We also need a condition for $a_\mathrm{osc}$. The transition to the oscillating phase happens when the
hubble friction term becomes negligible compared to the driving potential term
\begin{align}
    \label{eq:a_osc_cond}
    3 H(a_\mathrm{osc}) = m_a.
\end{align}
\begin{itemize}
    \item $a_\mathrm{osc} < a_\mathrm{eq}$ \\
    Here the universe is radiation dominated after $a_\mathrm{osc}$.
    Using condition \ref{eq:a_osc_cond} and equation \ref{eq:hubble_parameter_evo} we get
    \begin{align*}
        \left(\frac{m_a}{3}\right)^2 = H^2 = H_0^2 \left(
        \Omega_\mathrm{rad} \left( \frac{a_i}{a} \right)^4
    \right) \Rightarrow \frac{a_\mathrm{osc}}{a_0} = \left( \frac{9 \Omega_r H_0^2}{m_a^2} \right)^{1/4}
    \end{align*}
    And therefore
    \begin{align*}
        \Omega_a &= \frac{\rho_a}{\rho_c} = \frac{1}{\rho_c} \rho_a(a_\mathrm{osc}) \left( \frac{a_\mathrm{osc}}{a_0} \right)^3 \\
                 &= \frac{1}{3 H_0^2 M_\mathrm{pl}^2} \frac{1}{2} m_a^2 \phi_i^2 \left( \frac{9 H_0^2 \Omega_r}{m_a^2} \right)^{3/4} \\
                 &= \frac{1}{6} \left( 9 \Omega_r \right)^{3/4} \left( \frac{m_a}{H_0} \right)^{1/2} \left( \frac{\phi_i}{M_\mathrm{pl}} \right)^2.
    \end{align*}
    \item $a_\mathrm{osc} > a_\mathrm{eq} > 1$ \\
    Here the universe is matter dominated after $a_\mathrm{osc}$.
    Similar to the previous case we get
    \begin{align*}
        \Omega_a = \frac{9}{6} \Omega_\mathrm{mat} \left( \frac{\phi_i}{M_\mathrm{pl}} \right)^2
    \end{align*}
    \item $a_\mathrm{osc} \approx a_\mathrm{eq}$ \\
    Here we just get the density parameter at $a_\mathrm{osc}$
    \begin{align*}
        \Omega_a = \frac{1}{6} \left( \frac{m_a}{H_0} \right)^2 \left( \frac{\phi_i}{M_\mathrm{pl}} \right)^2
    \end{align*}
\end{itemize}

\subsection{QCD Axion and Temperature Dependent Mass}
\label{sec:temerature_dependent_mass}
In the case of the QCD axion the mass is actually temerature dependent.
The

\subsection{Anharmonic Effects}
\label{sec:anharmonic_effects}

\section{Further Steps}

\bibliography{Shared.bib}{}
\bibliographystyle{plain}

\end{document}
