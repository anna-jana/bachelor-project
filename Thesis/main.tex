\documentclass[a4paper, 12pt]{article}
% \usepackage[left=2.5cm, right=2.5cm,top=2.5cm,bottom=2.5cm]{geometry}
\usepackage[left=2.5cm, right=2.5cm,top=2.5cm,bottom=2.5cm]{geometry}

\usepackage[komastyle]{scrpage2}
\pagestyle{scrheadings}
\setheadsepline{0.5pt}[\color{black}]
% \usepackage[a4paper,margin=0.5in]{geometry}
%\setkomafont{captionlabel}{\sffamily\bfseries}
%\setcapindent{0em}
%\usepackage[ngerman]{babel}
\usepackage[utf8]{inputenc}
\usepackage{latexsym,exscale,stmaryrd,amssymb,amsmath}
\usepackage[nointegrals]{wasysym}
\usepackage{eurosym}
\usepackage{color}
\usepackage{rotating}
\usepackage{graphicx}
\usepackage{wrapfig}
% \usepackage{subfigure}
\usepackage{sidecap}
\usepackage{float}
\usepackage{hyperref}
\usepackage{subfig}
\usepackage{fancyvrb}
\usepackage{slashed}
\usepackage{setspace}
% \usepackage{subcaption}
%\usepackage{stackengine}
\usepackage{mathtools}
\usepackage{braket}
\usepackage[backend=bibtex, sorting=none]{biblatex}
\usepackage{csquotes}
\usepackage{dirtytalk}

\onehalfspace

\bibliography{Shared.bib}
% \DeclareMathOperator*{\diff}{d}
\newcommand{\diff}{\mathrm{d}}
\DeclareMathOperator{\trace}{Tr}

%\newcommand\stackapprox[2]{%
%  \mathrel{\stackunder[2pt]{\stackon[4pt]{\approx}{$\scriptscriptstyle#1$}}{%
%  $\scriptscriptstyle#2$}}}

\newcommand\myeq{\stackrel{\mathclap{\normalfont\mbox{def}}}{=}}

\title{Bachelor Thesis: \\ Computation of the Axion Relic Density}

\author{Janik Rieß
\\[0.5cm]{Advisor:
Prof. Dr. David J. E. Marsh}
}
\date{April/May 2019}

\begin{document}

\maketitle

\begin{abstract}
    In this thesis the density of Axion Dark Matter is computed. I present the full computation including the theoretical background and used methods.
    A fit for the numerical result was created, improving upon the parameters in analytic approximations given in the literature.
    Furthermore the developed code was used for an MCMC analysis where for QCD the most likely axion mass was found to be $m_a \approx ????? \pm ????????$.
    The same techique was applied to the Hidden Sector $\mu$-QCD model for Fuzzy Dark Matter, where the most likely Axion mass was found to be $m_a \approx ????? \pm ???????$.
\end{abstract}


\newpage
\tableofcontents
\newpage

\section{Introduction}
One of the big open questions in modern cosmology is the nature of dark matter (DM).
The axion solution is a standard model extension that was invented to solve the strong CP
problem and adds a new particle to the table, the (QCD) Axion.
Axion like particles (APL) also arise in string theory.
As a side effect, those new particles produce well motivated dark matter candidates.
The amount of axions in the universe today, the axion relic density, can be computed from two open parameters, the inital field value $\phi$ and the axion decay constant $f_a$.
Comparing those results with the observed amount of dark matter allows to exclude certain
parameter ranges and therefore help rule out axions as a dark matter model and narrow the search
for them.
In this report the building blocks for the computation of the axion relic density are presented.
First of all the equation of motion for the axion field is derived. Then the relic density
is analytically approximated for an axion like particle and for the QCD axion accordingly.


\section{Theory}
In this section the basics of cosmology are presented.
Only basic physics knowledge is assumed, as taught in an undergraduate
physics degree, but no derivations are given. The signature $(-, +, +, +)$ for the metric as well as Einstein sum
convention is used.

\subsection{Cosmology}
\noindent
In the field of Cosmology one tries to describe the universe as a whole
or a least on very large scales. Because one deals there with large distances
and masses, the framework of \emph{General Relativity} (GR) has to be used.
In GR space and time together are described by a \emph{Manifold}, that is a curved space.
Such a spacetime does not have a normal Minkovskian Metric, but a more complicated one.
On common assumption made in cosmology is that the universe
is homogeneous and isotropic, that is the universe is assumed to look the same
in every direction at every point in space. In this case the metric of the universe
is given as the \emph{Robinson-Walker metric} \cite[Eq. 2.1]{TheEarlyUniverseKolbAndTurner}
\begin{align}
    \label{eq:flrw}
    \diff^2 s = - \diff^2 t + a(t)^2 \left( \frac{\diff^2 r}{1 - k r^2} + \diff^2 \Omega \right)
\end{align}
Note that this metric contains an unknown parameter $a$. Since homogeneity was assumed,
it can not depend on the space coordinates but it can and will depend on time.
This parameter is called the \emph{scale parameter} since it describes in a way the size
of the universe. One defines $H := \frac{\dot{a}}{a}$
as the Hubble parameter.
There is also a parameter $k$, which fixes the topology of spacetime ie.\ its curvature.
For $k = +1$ the universe is a four-sphere, for $k = 0$ the universe is flat and for $k = -1$ the universe is a pseudo-sphere (an hyperbolic space) \cite[Sec. 3.1]{TheEarlyUniverseKolbAndTurner}.
Measurements have fixed the value of $k$ for our universe to be compatible with $k = 0$ and this value is used for the rest of the thesis.

\noindent
The second ingredient to GR is the content of the universe, that is matter, radiation etc.\ in the form of the \emph{energy-momentum tensor} $T_{\mu \nu}$.
It is defined as the tensor which gives in the $\mu, \nu$ component the flux of the $\mu$ component of the
momentum four-vector through a surface normal to the $x_\nu$ coordinate.
For a perfect fluid, that is a fluid which has no viscosity, shear stress or heat conduction,
the energy-momentum tensor has the form
\begin{align}
    T = \begin{pmatrix}
    \rho & j_x & j_y & j_z \\
    j_x & P & 0 & 0 \\
    j_y & 0 & P & 0 \\
    j_z & 0 & 0 & P
    \end{pmatrix}
\end{align}
where $P$ is the pressure, $\rho$ the energy density and $j_i$ the momentum density.
If the fluid is isotropic, all net fluxes and therefore momentum densities $j_i$ vanish. The momentum tensor then has a diagonal form \cite[Eq. 3.4]{TheEarlyUniverseKolbAndTurner}
\begin{align}
    \label{eq:energy_momentum_tensor_diag}
    T = \mathrm{diag} \, (\rho, P, P, P).
\end{align}

\noindent
The theory of GR now gives a connection between the geometry of spacetime,
the metric, and the content of spacetime, the energy tensor as the
Einstein field equations \cite[Eq. 3.1]{TheEarlyUniverseKolbAndTurner}
\begin{align}
    \label{eq:einstein_field_eq}
    R_{\mu \nu} - \frac{1}{2} R g_{\mu \nu} = 8 \pi G T_{\mu\nu}
\end{align}
where $R^{\mu \nu}$ is the \emph{Ricci tensor} and $R$ is the \emph{Ricci scalar}. Their are used to describe the
curvature of spacetime and are defined as \cite[Eq. 92.7]{Landau}
\begin{align}
    \label{eq:ricci}
    R_{\alpha \beta} &= \partial_\rho \Gamma^\rho_{\beta \alpha} - \partial_\beta \Gamma^\rho_{\rho \alpha} + \Gamma^\rho_{\rho \lambda} \Gamma^\lambda_{\beta \alpha} - \Gamma^\rho_{\beta \lambda} \Gamma^\lambda_{\rho \alpha} \\
    R &= R^\mu_\mu
\end{align}
using the \emph{Christoffel Symbols} \cite[Eq. 86.3]{Landau}
\begin{align}
    \Gamma^\alpha_{\beta \gamma} = \frac{1}{2} g^{\alpha \nu} \left( \partial_\gamma g_{\nu \beta} + \partial_\beta g_{\nu \gamma} - \partial_\nu g_{\beta \gamma} \right)
\end{align}
In general the left-hand-side of equation \ref{eq:einstein_field_eq} can contain a term $\Lambda g_{\mu \nu}$
where $\Lambda$ is cosmological constant, but it can be included as an additional
contribution to the energy-momentum tensor.
The physical content of equation \ref{eq:einstein_field_eq} was summarized
by John Archibald Wheeler as
\say{Spacetime tells matter how to move; matter tells spacetime how to curve} \cite[P. 5, left ]{Gravitation}.
Inserting the FRW metric into the Einstein field equations yields
the \emph{Friedmann equation} \cite[Eq. 3.10]{TheEarlyUniverseKolbAndTurner}
\begin{align}
    \label{eq:friedmann_eq}
    \frac{\dot{a}^2 + k}{a^2} = \frac{8 \pi G}{3} \rho
\end{align}
as well as
the \emph{acceleration equation} \cite[Eq. 3.11]{TheEarlyUniverseKolbAndTurner}
\begin{align}
    \label{eq:accelaration_eq}
    \frac{\ddot{a}}{a} = - \frac{4 \pi G}{3} \left( \rho + 3 P \right).
\end{align}
The continuity equation for the energy-momentum tensor can be written as
\begin{align}
    \label{eq:cont}
   \dot{\rho} = - 3 H \left( \rho + p \right).
\end{align}
The Friedmann eq.\ can also be written as \cite[Sec. 1.3.3, Eq. 1.67]{CosmologyBookMukhanov}
\begin{align}
    \label{eq:friedmann_equation}
    3 H^2 M^2_\mathrm{pl} = \rho,
\end{align}
where $M_\mathrm{pl} = \sqrt{\frac{1}{8 \pi G}}$
is the Planck mass.

\noindent
In order to solve those equations, one needs to find expressions
for the energy density $\rho$ and for the pressure $P$.
The simplest case for this is if the equation of state $w = P / \rho$ is constant.
Then one gets $\rho \propto a^{-3(w + 1)}$ \cite[Eq. 3.36]{TheEarlyUniverseKolbAndTurner} and $a \propto t^{2/3(w + 1)}$ if $k = 0$ \cite[Eq. 3.37]{TheEarlyUniverseKolbAndTurner}.
In cosmology one divides the content of the universe in several classes.
Particles who are non-relativistic are called matter or dust and have $w = 0$.
Particles who are ultra-relativistic are called radiation and have $w = 1/3$.
The case of the energy density being constant comes from the cosmological constant $\Lambda$ and has $w = -1$ as
the eq.\ of state. \cite[Eq. 3.7]{TheEarlyUniverseKolbAndTurner}
For those components the Friedmann equation can be written as
\cite[Eq. 9]{FriedmannPaper}
\begin{align}
    \label{eq:hubble_parameter_evo}
    H(t)^2 = H_0^2 \left(
        \Omega_\mathrm{rad} \left( \frac{a_i}{a} \right)^4 +
        \Omega_\mathrm{mat} \left( \frac{a_i}{a} \right)^3
    \right)
\end{align}
where $H_0$ is the Hubble parameter today, $a_i$ the scale factor today and the density parameter $\Omega = \frac{\varepsilon}{\varepsilon_c}$ (for each component) with $\varepsilon_c = 3 H_0^2 / M_\mathrm{pl}^2$ the critical density.
If $w$ is not constant is situation is more complicated. This happens for example if
some species of particle drop out of equilibrium with particle bath.
At this point the theory of thermodynamics comes into play. In general the system is described by a \emph{phasespace distribution function}  $f$, which is given by the
Einstein-Bose for Bosons or the Fermi-Dirac distributions for Fermions, if
the particles are in thermal and chemical equilibrium \cite[Sec. 3.3]{TheEarlyUniverseKolbAndTurner}. Then one computes
the required quantities ie.\ the pressure $P$, energy density $\rho$, the number density $n$ and the entropy density $s$ as expectation values
from $f(|\vec{p}|)$. For that one needs to approximate the resulting integrals or solve them numerically. For the ultra-relativistic case, one gets \cite[Eq. 3.51, 3.52, 3.59, 3.61, 3.62] {TheEarlyUniverseKolbAndTurner}
\begin{align}
    \label{eq:energy_denstiy}
    \rho =
   T^4 \sum_i \left( \frac{T_i}{T} \right)^4 \frac{g_i}{2 \pi^2} \int_{x_i}^\infty
   \frac{(u^2 - x_i^2)^{1/2} u^2 \diff u}{\exp(u - y_i) \pm 1} \approx
    \frac{\pi^2}{30} g_{\rho, *}(T) T^4
\end{align}
where $g_{\rho, *}$ is the effective number of relativitic degress for freedom and is given by
\begin{align}
    g_{\rho, *} = \sum_\mathrm{bosons} g_i \left( \frac{T}{T_i} \right)^4 + \frac{7}{8} \sum_\mathrm{ferminos} g_i \left( \frac{T}{T_i} \right)^4.
\end{align}
The number of internal degrees of freedom of a particle species is denoted by $g_i$ and its temperature by $T_i$ and uses $x_i := m_i / T$ as well as $y_i := \mu_i / T$.
Since the rest of the thesis deals with dynamics within the epoch of radiation domination, this energy density defines the
dynamics of the universe for the purpose of this work.
For the entropy density $s$ one finds a similar result \cite[Eq. 3.72, 3.73]{TheEarlyUniverseKolbAndTurner}
\begin{align}
    \label{eq:entropy_density}
    s = \frac{2 \pi^2}{45} g_{s, *} T^3 \, \, \mathrm{with} \, \,
     g_{s, *} = \sum_\mathrm{bosons} g_i \left( \frac{T}{T_i} \right)^3 + \frac{7}{8} \sum_\mathrm{ferminos} g_i \left( \frac{T}{T_i} \right)^3.
\end{align}


\noindent
\subsection{Axions}
One can measure the density parameter of matter today using eg.\
gravitational microlensing one finds that
only about $15\%$ of the matter in the universe are visible to us.
The remaining $85\%$ are called dark matter. The nature of dark matter is subject to ongoing research.

One model for dark matter is the axion model.
Axion are light pseudoscalar bosons that araise in both quantumchromodynamics (QCD)
as well as in string theory. Their have spin 0, that is their can be described
by a scalar field $\phi$.
Scalar Fields can be tough of as continues versions of a
lattice made out of springs. A cartoon of this picture can be found in figure ???????.
The action of the lattice in classical relativistic mechanics is given as
\begin{align}
    A = .....
\end{align}
By taking the limit $\Delta x \rightarrow 0$ one obtains
\begin{align}
    A = .....
\end{align}
Note that the Potential $V(\phi)$ is unspecified.

\subsubsection{Spontaneous Symmetry Breaking}
Where does the axion field come from?
Considering a Potential of the Form
\begin{align}
    V(\phi) =
\end{align}
for a complex scalar field $\phi$ and with the parameters $\lambda$ and $\nu$
one can see that the field has a unique vaccum state, that is the state of lowest energy, if ??????.
Otherwise the vaccum state is degenerated into a circle around the origin.
Since the potential obeys a $U(1)$ symmetry, the exact value of the vacuum state doesn't matter but is arbitrarily
chosen. One says that the symmetry is \emph{spontaneously broken}. If one taylor expands the field $\phi$ around the
vacuum state, new real valued scalar field emerge, so called (Pseudo-)Goldstone bosons \footnote{The prefix Pseudo-
refers to behavior under CP (Charge-Partiy) Symmetry) ??????}.
In general axions are such Pseudo-Goldstone bosons. Axions are initialy massless, but obtain their mass by couplings, but
not to the Higgsfield but via instantons effects. Instantons are pseudoparticles \footnote{Here the prefix pseudo
indicates that instantons are not elementary particles but appear in effective descriptions and behave like particles. ??????}
whose couplings generate not only a simple mass term but can also produce a more complicated potential.

\subsubsection{The Action}
Axions are described by a scalar field $\phi$. It is related to the theta parameter as $\theta = \phi / f_a$.
The action $S$ for a scalar field on a curved background is given after \cite[Chap. 4.1, Page 25]{MarshAxionCosmo}
 as
\begin{align}
    \label{eq:action}
    S[\phi] = \int \diff^4 x \sqrt{-g} \left(- \frac{1}{2} (\partial \phi)^2 - V(\phi) \right)
\end{align}
The measure $\diff^4 x \sqrt{-g}$ accounts for the effect of a curved background where $g$ is the determinant of the metric tensor $g_{\mu \nu}$ while
the Lagrange density describes a relativistic scalar field without any couplings to other fields
and a potential $V(\phi)$.
This potential contains both the mass term for the field
as well as  possible self interactions.
Because the axion mass is very small one
can assume that the axions form a condensate
that can be described by the classical equations
of motion so there is no need to quantize the
field theory, simplifying the computations
drastically. Analogously to how Maxwells
equations describe electromagnetic fields.

\subsubsection{The Klein Gordon Equation on a Curved Background}

The equation of motion is now derived from the action as given in equation \ref{eq:action} by varying the action by $\phi \rightarrow \phi + \delta \phi$.
The Variation is
\begin{align*}
    S[\phi + \delta \phi] &= \int \diff  x^4 \sqrt{-g} \left( - \frac{1}{2} (\partial (\phi + \delta \phi))^2 - V(\phi + \delta \phi)  \right) \\
    % &= \int \diff x^4 \sqrt{-g} \left( - \frac{1}{2} g^{\mu \nu} (\partial_\mu \delta \phi \partial_\nu \phi + \partial_\nu \delta \phi \partial_\mu \phi) - V'(\phi) \delta \phi \right) \\
    &= - \frac{1}{2} \left( \int \diff x^4 \sqrt{-g} g^{\mu \nu} \partial_\mu \delta \phi \partial_\nu \phi + \int \diff x^4 \sqrt{-g} g^{\mu \nu} \partial_\nu \delta \phi \partial_\mu \phi \right) - \int \diff  x^4 \sqrt{-g} V'(\phi) \delta \phi \\
    % &= - \frac{1}{2} \left( - \int \diff x^4 \delta \phi \partial_\mu \sqrt{-g} g^{\mu \nu} \partial_\nu \phi +
    %                        - \int \diff x^4 \delta \phi \partial_\nu \sqrt{-g} g^{\mu \nu} \partial_\mu \phi \right) - \int \diff  x^4 \sqrt{-g} V'(\phi) \delta \phi \\
    &= \int \diff  x^4 \delta \phi \left( \partial_\mu \sqrt{-g} g^{\mu \nu} \partial_\nu \phi - \sqrt{-g} V'(\phi) \right)
\end{align*}
using
$
% \begin{align*}
    V(\phi + \delta \phi) \approx V(\phi) + \frac{\partial V(\phi)}{\partial \phi} \delta \phi + O(\delta \phi^2)
% \end{align*}
$
and
$(\partial (\phi + \delta \phi))^2  = g^{\mu \nu} (\partial_\nu \phi \partial_\mu \phi + \partial_\nu \phi \partial_\mu \delta \phi + \partial_\nu \delta \phi \partial_\mu \phi + \partial_\nu \delta \phi \partial_\mu \delta \phi) $
% \begin{align*}
      % (\partial (\phi + \delta \phi))^2
                                        % &= (\partial^\mu (\phi + \delta \phi)) (\partial_\mu (\phi + \delta \phi)) \\
                                        % &= g^{\mu \nu} (\partial_\nu (\phi + \delta \phi)) (\partial_\mu (\phi + \delta \phi)) \\
       %                                 &= g^{\mu \nu} (\partial_\nu \phi \partial_\mu \phi + \partial_\nu \phi \partial_\mu \delta \phi +
       %                                                 \partial_\nu \delta \phi \partial_\mu \phi + \partial_\nu \delta \phi \partial_\mu \delta \phi) % \\
                                        % &= \mathrm{const} + g^{\mu \nu} (\partial_\nu \phi \partial_\mu \delta \phi + \partial_\nu \delta \phi \partial_\mu \phi) + O(\delta \phi ^2)
% \end{align*}
as well was integration by parts % in the forth equal sign
and the symmetry of $g^{\mu \nu}$. % and renaming indicies in the last
% equal sign.
The term linear in $\delta \phi$ has to vanish in order to fulfill Hamiltons principle so
\begin{align}
    \label{eq:klein_gordon}
    \Box \phi = \frac{1}{\sqrt{-g}} \partial_\mu \sqrt{-g} g^{\mu \nu} \partial_\nu \phi = \frac{\partial V(\phi)}{\partial \phi}
\end{align}
holds.
This is the Klein Gordon Equation for the axion field. \cite[Chap. 4.1, Page 26]{MarshAxionCosmo}


\subsubsection{Equation for the Background Field}
In order to compute the relic density one is interested in the evolution of the uniform background field $\bar{\phi}$ with $\phi = \bar{\phi} + \delta \phi$, where
$|\bar{\phi}| \gg |\delta \phi|$ and $\nabla \bar{\phi} = 0$ (uniformity). One therefore only has to consider the time dependence in equation \ref{eq:klein_gordon}. From now on $\phi$ will label the background field.
For the background field all spacial derivatives $\partial_i$ vanish, therefore one can drop them in equation \ref{eq:klein_gordon} from the summation and only the time derivative remains.
\begin{align*}
    \frac{\partial V}{\partial \phi} &= \Box \phi
                                     % &= \frac{1}{\sqrt{-g}} \partial_t \sqrt{-g} g^{00} \partial_t \phi \\
                                     = \frac{1}{\sqrt{-g}} \left( (\partial_t \sqrt{-g} g^{00}) (\partial_t \phi) + \sqrt{-g} g^{00} \partial^2_t \phi \right)
\end{align*}
For the background field only the FLRW metric \ref{eq:flrw}
is important since spacial dependence is ignored and the universe is therefore homogeneous and isotropic. Its determinant is
% \begin{align*}
%     g^{00} = -1
% \end{align*}
\begin{align*}
    g = \det g_{\mu \nu} = -1 \cdot \frac{a(t)^2}{1 - kr^2} \cdot a(t)^2 r^2 \cdot a(t)^2 r^2 \sin^2 \theta = - a(t)^6 f(\vec{x}),
\end{align*}
where $f$ is some function of the spatial coordinates. One computes:
\begin{align*}
    \frac{\partial V}{\partial \phi} &= - \frac{1}{\sqrt{a(t)^6 f(\vec{x})}} (\partial_t \sqrt{ a(t)^6 f(\vec{x}) }) \dot{\phi} - \ddot{\phi}
                                     %= - \frac{\partial_t a(t)^3}{a(t)^3} \dot{\phi} - \ddot{\phi} \\
                                     %&= - \frac{\dot{a}(t) \cdot 3 a(t)^2}{a(t)^3} \dot{\phi} - \ddot{\phi}
                                     %= - 3 \frac{\dot{a}}{a} \dot{\phi} - \ddot{\phi}
                                     = - 3H \dot{\phi} - \ddot{\phi}.
\end{align*}
This yields the equation of motion for the axion field \cite[Chap 4.2, Page 25]{MarshAxionCosmo}
\begin{align}
    \label{eq:eom}
    \ddot{\phi} + 3 H \dot{\phi} + V'(\phi) = 0
\end{align}
The energy density $\rho_a$ and the pressure $P_a$ of the axion field
can be computed from the stress energy tensor
$T_{\mu \nu} = -2 \frac{\partial \mathcal{L}}{\partial g_{\mu \nu}} + g_{\mu \nu} \mathcal{L}$
\cite[Sec. 3.3, Page 161, Eq. 3.21]{ClassicalFieldTheory}
, resulting in \cite[Sec. 4.2, Page. 25]{MarshAxionCosmo}
\begin{align}
    \label{eq:axion_energy_density_and_pressure}
    \notag \rho_a &= T_{00} % = - 2 \frac{\partial}{\partial g^{00}} \left( -1 \frac{1}{2}(\partial_\mu \phi)(\partial_\nu \phi) - V(\phi))
     %\right) + g_{00} \left( -1 \frac{1}{2}(\partial_\mu \phi)(\partial_\nu \phi) - V(\phi))
     %\right) \\
     %&= -2 \left(- \frac{1}{2} (\partial_0 \phi)^2 \right) + (-1) \left(- \frac{1}{2} (-1) (\partial_0 \phi)^2 - V(\phi)\right)
     =  \frac{1}{2} \dot{\phi}^2 + V(\phi) \\
    P_a &= T_{11} % = \frac{1}{2} \phi^2 - g_{11} V(\phi)
    = \frac{1}{2} \dot{\phi}^2 - V(\phi)
\end{align}

\subsection{The Potential}
The potential $V(\phi)$ is unknown at this point for a general axion model, but it has a minimum
at $\phi = 0$. Using the Taylor expansion of V about $\phi = 0$, the potential is approximately given as
\begin{align}
    \label{eq:potential}
    V(\phi) = \frac{1}{2} m_a^2 \phi^2
\end{align}
where $m_a$ is the axion mass as the coupling constant.
There are two remarks to be made.
First, there might be higher order or anharmonic effects.
The potential for the QCD axion is an example for this, as will be discussed in section \ref{sec:temerature_dependent_mass}.
Second, the axion mass can depend on the temperature of the universe like in the case of the QCD axion.

The mass of the axion depends on the temperature in the case of the QCD axion, because its rest mass is not due
to the coupling to the Higgs field but because of the dependence of the vacuum energy of the axion field
via instanton effects.
This effect is covered in section \ref{sec:temerature_dependent_mass}.
The axion mass for a general model is left open as a parameter to be tuned to the observed
natural world.

% For the QCD axion the potential can be computed \cite[Sec. 3, Page 5]{AxionCosmoRev} and is approximatly given as
% \begin{align}
%     \label{eq:qcd_axion_potential}
%     V(\phi) = \frac{m_a(T)^2}{f_a^2} (1 - \cos(\phi)).
% \end{align}
% So
% \begin{align*}
%     V'(\phi) = \frac{m_a(T)^2}{f_a^2} \sin(\phi)
% \end{align*}
% The quation of motion for this potential has to be solved numerically.

\subsection{Initial Conditions}
\label{sec:initial_conditions}
Since the equation of motion in formula \ref{eq:eom} is a second order ODE, one needs the initial values
for $\phi$ and $\dot{\phi}$.
Because $m_a \ll H(t_i)$ the friction term $\sim \dot{\phi}$ in equation \ref{eq:eom} dominates over the
driving force $V'(\phi) \sim m_a^2$ from the potential term at early times.
It can be assumed that $\dot{\phi_i} = 0$.
The initial condition for $\phi_i$ is left as an open parameter that can be tuned to
reach the correct density of DM for today \cite[Chap. 4.2, Page 26]{MarshAxionCosmo},
like the axion mass $m_a$.
The later results depend on the square of the initial field value, therefore the fluctuations $\sigma_\phi$ can contribute to the
inital value
\begin{align}
    \langle \phi_i^2 \rangle = \langle \phi_i \rangle ^ 2 + \sigma_\phi ^ 2
\end{align}
But since only $\langle \phi^2_i \rangle$ is needed
it can be taken as a free parameter as long as the spacial  dependence is ignored.

\section{Axion Like Particles}
% In the following section an analytic solution for the equation of motion is discussed.
In the following section a general model for an axion like particle is presented. Those calculations also hold
for eg. a axion model from string theory.
An analytic solution for a special case is given and the general scaling behavior during the later evolution of the field is derived.
For this a harmonic potential as well as a constant axion mass are assumed.

\subsection{Analytic Solution for Scale Parameter as a Power Law} % $a \sim t^p$}
If the time dependence of the scale parameter $a(t)$ is a power law $a(t) \sim t^p$ like for radiation $p = \frac{1}{2}$
or matter domination $p = \frac{3}{4}$ \cite[Sec. 3.2, Page 58]{TheEarlyUniverseKolbAndTurner}
the equation \ref{eq:eom} is analytically solvable \cite[Chap. 4.2, Page 25]{MarshAxionCosmo}.
One uses the Ansatz
\begin{align*}
    \phi(t) &= \kappa \cdot t^m \cdot f(t)
    % \dot{\phi}(t) &= \kappa \left( m t^{m - 1} f + t^m \dot{f} \right) \\
    % \ddot{\phi}(t) &= \kappa \left( m (m - 1) t^{m - 2} f + 2 m t^{m - 1} \dot{f} + t^m \ddot{f} \right)
\end{align*}
for some constant $m$ and some function $f(t)$ and inserts it into equation \ref{eq:eom} yielding % , dividing out $\kappa$ and gets
\begin{align*}
    % &\Rightarrow m (m - 1) t^{m - 2} f + 2 m t^{m - 1} \dot{f} + t^m \ddot{f} + 3H \left( m t^{m - 1} f + t^m \dot{f} \right) + m_a^2 t^m f = 0 \\
    % &\Rightarrow m (m - 1) t^{- 2} f + 2 m t^{- 1} \dot{f} + \ddot{f} + 3H \left( m t^{- 1} f + t^m \dot{f} \right) + m_a^2 f = 0 \\
    % &\Rightarrow f \left( m (m - 1) t^{-2} + 3H m t^{-1} + m_a^2 \right) + \dot{f} \left( 2mt^{-1} + 3 H \right) + \ddot{f} = 0 \\
    % &\Rightarrow f \left( m (m - 1) + 3H m t + t^2 m_a^2 \right) + \dot{f} \left( 2mt + 3 H t^2 \right) + t^2 \ddot{f} = 0 \\
    % &\Rightarrow f \left( m (m - 1) + 3pm + t^2 m_a^2 \right) + \dot{f} \left( 2mt + 3 t p \right) + t^2 \ddot{f} = 0 \\
    % &\Rightarrow
    f \left( m (m - 1) + 3pm + t^2 m_a^2 \right) + t \dot{f} \left( 2m + 3 p \right) + t^2 \ddot{f} = 0.
\end{align*}
If for the exponent $m$
%\begin{align*}
$
    2m + 3p = 1 \iff m = - 3 / 2 p + 1 / 2
$
%\end{align*}
holds, then using
%\begin{align*}
$
    m (m - 1) + 3pm % &= (-3/2p + 1/2) (-3/2p - 1/2) + 3p(-3/2p + 1/2) \\
                    % &= - (3p)^2 / 4 + p/2 - (1/2)^2
                    = - \left(\frac{3p + 1}{2}\right)^2 =: -n^2
$
%\end{align*}
one gets
\begin{align*}
    (t^2 m_a^2 - n^2) f + t \dot{f} + t^2 \ddot{f} = 0.
\end{align*}
% for some function $f$.
This equation can be recognized as the Bessel equation and therefore the function $f$ is a linear combination of Bessel functions.
So the solution for the equation of motion in this case is
\begin{align}
    \label{eq:power_law_solution}
    \phi(t) = a^{-3/2} \left(\frac{t}{t_i}\right)^{1/2}\left(C_1 J_n(m_a t) + C_2 Y_n(m_a t)\right)
\end{align}
for $\phi(a_i) = \phi(t_i) = \phi_i$
where $J_n$ and $Y_n$ are the Bessel function of first and second kind with $n = (3p - 1) / 2$.
The constants $C_1$ and $C_2$ are fixed by the initial conditions from section \ref{sec:initial_conditions}.
Plugging in the solution yields
\begin{align*}
    C_1 &= \frac{B \phi_i}{BC - AD}, \, \, \,
    C_2 = \frac{A \phi_i}{BC - AD} \\
    A &= \alpha J_n(m_a t_i) + m_a J_n'(m_a t_i), \, \, \,
    B = \alpha Y_n(m_a t_i) + m_a Y_n'(m_a t_i) \\
    C &= a_i^{-3/2} J_n(m_a t_i), \, \, \,
    D = a_i^{-3/2} Y_n(m_a t_i) \\
    \alpha &= \left(-\frac{3}{2} p + \frac{1}{2}\right) t_i^{-1}
\end{align*}
The solution was plotted for arbitrary values in figure \ref{fig:rad_dom_ax_field} for the radiation dominated case.
\begin{figure}[H]
    \centering
    \includegraphics[width=0.7\linewidth]{analytic_power_law_plot.pdf}
    \caption{Evolution of the axion field in a radiation dominated universe. This plot is a replication of figure 4 in \cite{MarshAxionCosmo}. The dashed line marks the point of $a_\mathrm{osc}$.
    In plot a), the axion field is plotted against the normalized scalefactor. The condition for the oscillation is shown in b).
    In c) one can see the behavior of the equation of state and in d) the energy density including the result using the
    WKB approximation in section \ref{sec:wkb_alp} }
    \label{fig:rad_dom_ax_field}.
\end{figure}

\subsection{WKB Approximation}
\label{sec:wkb_alp}

As one can see in figure \ref{fig:rad_dom_ax_field} plot a), the axion field first starts with an almost constant value, since $\dot{\phi}_i = 0$ and starts to oscillate
at some scale factor $a_\mathrm{osc}$. After this point the energy density of the axion field seems to fall off
described by a power law. To quantify this observation one makes the Ansatz \cite[Chap 4.3.1, Page 28]{MarshAxionCosmo}
% Furthermore we can use the WKB approximation to gain a better understanding of the formation of the relic density
\begin{align*}
    \phi &= A c \, \, \Rightarrow \,\,
    \dot{\phi} = \dot{A} c - s m_a A, \, \,
    \ddot{\phi} = \ddot{A} c - 2 m_a \dot{A} s - m_a^2 c A
\end{align*}
where $A = A(t)$ is a slowly varying function of the time $t$. With the properties of the axion field during the oscillation $H / m_a \sim \dot{A} / m_a \sim \epsilon \ll 1$ and $\ddot{A} \ll 1$
and $c = \cos(m_a t + \theta)$, $s = \sin(m_a t + \theta) $
and plug it into equation \ref{eq:eom}
resulting in
\begin{align*}
    &\ddot{A} c - 2 m_a \dot{A} s - m_a^2 c A + 3 H ( \dot{A} c - m_a s A) + m_a^2 c A = 0
    &\Rightarrow \underbrace{\frac{c}{m_a^2} \ddot{A}}_{\approx 0} - 2 s \underbrace{\frac{\dot{A}}{m_a}}_{= \epsilon} + 3c \underbrace{\frac{H \dot{A}}{m_a^2}}_{= \epsilon^2} - 3 s A \underbrace{\frac{H}{m_a}}_{= \epsilon} = 0
\end{align*}
using the properties of $A(t)$.
This gives
\begin{align*}
    \frac{\dot{A}}{A} = - \frac{3}{2} H = - \frac{3}{2} \frac{\dot{a}}{a}
\end{align*}
in leading order.
Using the Ansatz for $A = a^n$ we get
\begin{align*}
    \frac{n \dot{a} a^{n - 1}}{a^n} = - \frac{3}{2} \frac{\dot{a}}{a} \Rightarrow n = - \frac{3}{2}
\end{align*}
So it can be seen that the axion field indeed scales at later times like a power law $\sim a^n = a^{-3/2}$
and oscillates with a frequency of $\omega = m_a$ as can be seen from
the cosine term. Therefore, the energy density oscillates with $\omega = 2 m_a$
since it is quadratic in the field and
the expectation value of the equation of state $w = \frac{P_a}{\rho_a} \approx 0$ and the axion density goes by
\begin{align*}
    \rho_a = \frac{1}{2}m_a^2\phi^2 \sim \left(a^{-3/2}\right)^2 = a^{-3}
\end{align*}
in this period. Therefore, the axion fields behaves like ordinary matter at this point and
is just diluted by the expansion of the universe, resulting in \cite[Chap. 4.3.1, Page 28]{MarshAxionCosmo}
\begin{align}
    \label{eq:wkb_axion_denity}
    \rho_a(a) = \rho_a(a_\mathrm{osc}) \left( \frac{a_\mathrm{osc}}{a} \right)^3
\end{align}
One can use this to derive analytic formulas for the axion density parameter for some cases \cite[Chap 4.3.1, Page 28]{MarshAxionCosmo}
For this one needs a condition for $a_\mathrm{osc}$ eg. the time where the WKB approximation can be used, that is the time where the field starts to
oscillate.
The oscillating phase happens when the Hubble friction term becomes negligible compared to the driving potential term,
so the transition is at
\begin{align}
    \label{eq:a_osc_cond}
    3 H(a_\mathrm{osc}) = m_a.
\end{align}
In general any constant factor can be multiplied to this expression, like in figure \ref{fig:rad_dom_ax_field} the factor  $3$
was replaced with a $2$, because this only gives a rough estimate that can be adjusted.
Now consider the following cases for the onset of oscillation, where the oscillation phase
begins before or after the point of matter radiation equality $a_\mathrm{eq}$ where $\rho_\mathrm{rad} = \rho_\mathrm{mat}$.
\begin{itemize}
    \item $a_\mathrm{osc} < a_\mathrm{eq}$ \\
    Here the universe is radiation dominated after $a_\mathrm{osc}$.
    Using condition \ref{eq:a_osc_cond} and equation \ref{eq:hubble_parameter_evo} we get
    \begin{align*}
        \left(\frac{m_a}{3}\right)^2 = H^2 = H_0^2 \left(
        \Omega_\mathrm{rad} \left( \frac{a_i}{a} \right)^4
    \right) \Rightarrow \frac{a_\mathrm{osc}}{a_0} = \left( \frac{9 \Omega_r H_0^2}{m_a^2} \right)^{1/4}
    \end{align*}
    And therefore
    \begin{align*}
        \Omega_a &= \frac{\rho_a}{\rho_c} = \frac{1}{\rho_c} \rho_a(a_\mathrm{osc}) \left( \frac{a_\mathrm{osc}}{a_0} \right)^3
                 &= \frac{1}{3 H_0^2 M_\mathrm{pl}^2} \frac{1}{2} m_a^2 \phi_i^2 \left( \frac{9 H_0^2 \Omega_r}{m_a^2} \right)^{3/4}
                 &= \frac{1}{6} \left( 9 \Omega_r \right)^{3/4} \left( \frac{m_a}{H_0} \right)^{1/2} \left( \frac{\phi_i}{M_\mathrm{pl}} \right)^2.
    \end{align*}
    \item $a_\mathrm{osc} > a_\mathrm{eq} > 1$ \\
    Here the universe is matter dominated after $a_\mathrm{osc}$, similar to the previous case
    \begin{align*}
        \Omega_a = \frac{9}{6} \Omega_\mathrm{mat} \left( \frac{\phi_i}{M_\mathrm{pl}} \right)^2
    \end{align*}
    \item $a_\mathrm{osc} \approx 1$ \\
    Here one just gets the density parameter at $a_\mathrm{osc}$
    \begin{align*}
        \Omega_a = \frac{1}{6} \left( \frac{m_a}{H_0} \right)^2 \left( \frac{\phi_i}{M_\mathrm{pl}} \right)^2
    \end{align*}
\end{itemize}
A contour plot for the first case $a_\mathrm{osc} < a_\mathrm{eq}$ can be found in figure \ref{fig:wkb_ard} showing the computed density parameter as a function of both free parameters
initial field value and axion mass. The area where $\Omega_a$ is greater than $\Omega_\mathrm{DM}$ is the region that is excluded for the axion parameters to be in,
since that would contradict the observed amount of dark matter. But it is entirely possible that $\Omega_a < \Omega_\mathrm{DM}$ and not all the dark matter
is made out of axions.
\begin{figure}
    \centering
    \includegraphics[width=0.7\linewidth]{pow_law_wkb_plot.pdf}
    \caption{Axion Relic Density using a WKB approximation for $m_a = \mathrm{const}$
    as a function of the initial field value $\phi_i$ and the axion mass $m_a$. The white area is exclude for the axion $\Omega_a > \Omega_\mathrm{DM}$.
    }
    \label{fig:wkb_ard}
\end{figure}

\section{QCD Axion and Temperature Dependent Mass}
\label{sec:temerature_dependent_mass}
In the previous section the temperature dependence of the axion mass was ignored but
in case of the QCD axion the temperature dependence cannot be ignored \cite[Sec. 4.3.2, Page 30] {MarshAxionCosmo}.
This also means that an approximation as presented in section \ref{sec:wkb_alp} of the relic density is not possible, since the derivation of
the WKB approximation assumed $m_a = \mathrm{const}$.
In this section an alternative way of scaling the axion density from its value at $a_\mathrm{osc}$ until today is derived.
Then different analytic results, derived in the literature, are presented and one of them then used to
compute the relic density for the QCD axion approximately.

\subsection{Conservation of Axion Number and Entropy}
\label{sec:thermo}
In order to do a similar calculation like in section
\ref{sec:wkb_alp},
one needs to show that
also for the QCD axion the number of axions is approximately conserved at some point
in its evolution. In order to scale the density to the present one also needs to take
the expansion of the universe into account. This can be done
by using the entropy conservation during the expansion.
If both of those quantities are conserved,
this allows to scale the relic density to the present by
\begin{align}
    \label{eq:n_over_s}
    \frac{n}{s} = \frac{N / a^3}{S / a^3} = \frac{N}{S} = \mathrm{const}
    \Rightarrow n_a(\mathrm{today}) = s(\mathrm{today}) \frac{n_a(T_\mathrm{osc})}{s(T_\mathrm{osc})}.
\end{align}
First, lets show that the axion number is conserved.
This derivation follows \cite[Sec. IV, Eq. 28]{AxionCosmoRev}.
For this, the derivative of the axion energy density from equation \ref{eq:axion_energy_density_and_pressure} is computed as
\begin{align*}
    \dot{\rho} &= \dot{\theta} \ddot{\theta} + 2 m_a \dot{m_a} (1 - \cos \theta) + m_a^2 \dot{\theta} \sin \theta
    % = \dot{\theta}(-3 H \dot{\theta} - m_a^2 \sin \theta) + 2 m_a \dot{m_a} (1 - \cos \theta) + m_a^2 \dot{\theta} \sin \theta
    = 2 m_a \dot{m_a} (1 - \cos \theta) - 3 H \dot{\theta}^2,
\end{align*}
using the equation of motion \ref{eq:eom}.
Note that in this computation the axion field is described by $\theta = \phi / f_a$
Now one looks at the later times in the evolution of the axion field, when the axion
is oscillating.
For this it is assumed
that the axion mass $m_a$ and the hubble parameter $H$ change slowly with time compared to the oscillation of the axion field. Therefore the time dependence of the coefficients
in the field equation can be neglected over the coarse of one oscillation.
Also the average over one oscillation is used, where an overline
is used to indicate the mean value over one oscillation.
The need to consider the more complicated cosine potential for the QCD axion is
only due to the early evolution, when $\phi$ is not close to zero but
in the oscillating epoch the approximation $\phi \ll 1 \Rightarrow V(\theta) \approx m_a^2 \theta_a^2$ can be used.
Then the equation of motion behaves like a damped oscillator with constant coefficients and the
virial theorem holds for the mean kinetic energy $K = \overline{\dot{\theta}^2} / 2$ and the mean potential energy $V = m_a(T)^2 \overline{\theta^2}$ as
\begin{align*}
    2K + V = 0 \Rightarrow \overline{\dot{\theta}^2} = m_a^2 \overline{\theta^2}.
\end{align*}
Then the time derivative of the axion energy density can be written as
\begin{align*}
    \overline{\dot{\rho}} \approx m_a \dot{m_a} \overline{\theta^2} - 3 H m_a^2 \overline{\theta^2}.
\end{align*}
Now considering the axion number in a comoving volume of space
\begin{align*}
    N = \frac{a^3 \rho_a}{m_a}
\end{align*}
its time derivative can be computed as
\begin{align*}
    \dot{N} &= \frac{a^3}{m_a} (m_a \dot{m_a} \overline{\theta^2} - 3 H m_a^2 \overline{\theta^2}) + \frac{3 \dot{a} a^2}{m_a} \left( \frac{1}{2} \overline{\dot{\theta}^2} + \frac{1}{2} m_a^2 \overline{\theta^2} \right)
    % &= \frac{a^3}{m_a}( m_a \dot{m_a} \overline{\theta^2} - 3 \frac{\dot{a}}{a} m_a^2 \overline{\theta^2}) + \frac{3 \dot{a} a^2}{m_a} m_a^2 \overline{\theta^2} \\
    = a^3 \dot{m_a} \overline{\theta^2} \approx 0.
\end{align*}
Therefore the axion number is conserved.


Second for the entropy conservation, we need that the universe is
in local equilibrium in order to used thermodynamics. The proof from \cite[Sec. 3.3, Page. 74]{CosmologyBookMukhanov} is used and presented there.
For a plasma to be in thermal equilibrium the expansion timescale $\tau_H$ has
to be much larger than the timescale for collisions in the plasma $\tau_\mathrm{col}$.
The expansion timescale is given from the Hubble parameter as
\begin{align*}
    \tau_H = \frac{1}{H} = \left( \frac{\pi g_{*,s} T^2}{\sqrt{90} M_\mathrm{pl}} \right)^{-1}
    \sim \frac{1}{T^2}
\end{align*}
using the Friedmann equation \ref{eq:friedmann_equation} and the
energy density in equation \ref{eq:energy_denstiy} for the thermal case.
The collision timescale can be estimated from the crosssection
$\sigma \sim \alpha^2 \lambda_c^2 \sim \alpha \frac{1}{T^2} \sim \frac{1}{T^2}$
using the Compton wavelength $\lambda_c \sim 1 / p \sim 1 / E \sim 1 / T$
and the number density $n \sim T^3$ as well as the velocity
$v \sim 1$ since all particles are ultra relativistic as
\begin{align*}
    % \tau_c = \frac{1}{\sigma n v} \sim \frac{1}{T^{-2} T^3} = \frac{1}{T}
    \tau_c = \frac{1}{\sigma n v} = \frac{1}{T}
\end{align*}
Therefore
\begin{align*}
    T^{-2} \gg T^{-1} \Rightarrow \tau_H \gg \tau_c
\end{align*}
if $T$ is high enough and then the early universe is in
local equilibrium.
Since this argument is using the formulas for
the number density, energy density and velocity
as a function of temperature it has a circular structure to it. If the universe would not be
in local thermal equilibrium
there wouldn't even be a well defined  temperature!
But this argument can be seen as a proof
that once the universe is in local equilibrium
it stays in local equilibrium.

Once we have established the local equilibrium
we can proof the conservation of
entropy following \cite[Sec. 3.4, from page 65]{TheEarlyUniverseKolbAndTurner}.
Using the first law of thermodynamics (neglecting the chemical potential compared to the temperature) we get
\begin{align}
    \label{eq:first_law}
    \diff E &= T \diff S - P \diff V
    \Rightarrow \diff (\rho V) = T \diff S - P \diff V
    \Rightarrow T \diff S%\diff (\rho V) + P \diff V \\ &= \diff (PV) - \diff (PV) + \diff (\rho V) + P \diff V
    %= \diff (PV) - \diff (PV) + \diff (\rho V) + P \diff V
    = \diff ((P + \rho) V) - V \diff P.
\end{align}
for a co-moving volume $V$, energy density $\rho$, pressure $P$ and entropy $S$ in this comoving volume.
This leads to
\begin{align*}
    \diff S = \frac{P + \rho}{T} \diff V + V \diff (P + \rho) - V \diff P
            = \frac{\partial S}{\partial V} \diff V + V \diff \rho
\end{align*}
Using a maxwell relation one obtains
\begin{align*}
    \frac{\partial P}{\partial T} = - \frac{\partial^2 F}{\partial T \partial V} = \frac{\partial S}{\partial V} = \frac{P + \rho}{T}
\end{align*}
where $F$ is the free energy and therefore
\begin{align}
    \label{eq:dP}
    \diff P = \frac{\rho + P}{T} \diff T.
\end{align}
Using equation \ref{eq:first_law} and \ref{eq:dP}  $\diff S$ is given by
\begin{align}
    \diff S = \frac{\diff((P + \rho)V)}{T} - \frac{V\left(\frac{\rho + P}{T}\right) \diff T}{T}
    = \diff \left( \frac{V(P + \rho)}{T} + \mathrm{const} \right)
\end{align}
From again the first law of thermodynamics we get for an adiabatic process ($\diff Q = 0$)
\footnote{A homogeneous universe can't exchange heat with another system.}
\begin{align*}
    0 = \diff E = \diff A = - P \diff V = - (\diff (P V) - V \diff P)
    \Rightarrow \diff ((\rho + P) V) = V \diff P = \frac{V(P + \rho)}{T} \diff T
    \Rightarrow \diff \left( \frac{V(P + \rho)}{T} \right) = 0
\end{align*}
and therefore entropy is conserved.

\subsection{Axion Potential in QCD}
In order to compute the relic density for the QCD axion, its potential $V(\phi)$ has to be known, that is
both the dependence on $\phi$ or $\theta$ as well as the dependence on the temperature $T$.
In this section an overview over the derivation of the axion mass at $T = 0$ using chiral pertubation theory as well as the potential at
high temperatures in the dilute instanton gas approximation is given.
Those results can be used together as the axion potential by using the
$\theta$ dependence from the high temperature and the zero temperature
mass once the high temperature result would exceed the zero temperature mass.
The exact calculations and their technical details are beyond
the scope of this report \footnote{and the abilities of the author}.
% \subsubsection{The theta term in QCD}
The general $\mathrm{SU}(3)$ symmetric QCD lagrangian is \cite[Chap. VII.3, Page 369]{Nutshell}
\begin{align*}
    \mathcal{L}_\mathrm{QCD} = - \frac{1}{4} G^a_{\mu \nu} G^{\mu \nu}_a - \overline{q} \left( i \slashed{D} - \mathcal{M} \right) q + \underbrace{i \frac{N_f g^2 \theta}{32 \pi^2} G_{\mu \nu}^a \tilde{G}_{\mu \nu}^a}_{= \mathcal{L}_\theta}
\end{align*}
where $G_{\mu \nu}$ is the Gluon field strength tensor with its dual $\tilde{G}_{\mu \nu}$, $q$ is a vector of quark spinors,
$\slashed{D}$ the covariant derivative, $\mathcal{M}$ the quark mass matrix,
$N_f$ the number of fermions eg. quarks, $g$ the coupling strength and $\theta$ is a free parameter modulo $2\pi$.
The term $\mathcal{L}_\theta$ is the so called theta term of QCD. It can be written as a total derivative and therefore has no effect to
any pertubative approximation eg. it does not influence the classical equations of motion nor any Feynman diagram.
It is a topological invariant eg. the integral
\begin{align*}
    \int \diff^4 x \mathcal{L}_\theta = i \theta Q
\end{align*}
is a winding number $Q \in \mathbb{Z}$ \cite[Sec. II, Eq. 2.6]{Leutwyler:1992yt}. The quantity $Q$ is called the topological charge.
Nevertheless the theta term has non pertubative effects and violates CP symmetry, like giving an electric dipole moment to the neutron. Since no such dipole moment was measured, the free parameter $\theta$
has to be very close to zero. Why $\theta$ is so small is called the strong CP problem and gives rise to the axion idea in the first place. The basic idea is that $\theta$ is not a free parameter
but a dynamical field, which, using the theta term, couples to the gluon field. Therefore
the energy of the vacuum state of QCD has a dependence on $\theta$ and the energy of the QCD vacuum
is a potential energy for the axion field. This dependence has to be computed.
The vacuum energy, normalized to be zero at $\theta = 0$, is given as
\begin{align*}
   \epsilon_0(\theta) = - \frac{1}{V} \log \frac{Z(\theta)}{Z(0)}
\end{align*}
where $V$ is the considered four volume and $Z$ the partition function of the QCD vacuum state
\cite[Eq. 2.2]{FiniteTempQCD}.
Note that this normalization is an arbitrary choice.
That the potential $\epsilon(\theta)$ has its minimum at $\theta = 0$ is stated by the Vafa and Witten theorem \cite{VafaWitten}.
The vacuum state in QCD is not unique, since for every possible topological charge $Q$
there is a different topology and therefore a different vacuum state.
The QCD vacuum as a hole can then be written as a superposition
\begin{align*}
    |\theta \rangle  = \sum_{Q = - \infty}^{\infty} e^{i \theta Q} |Q \rangle.
\end{align*}
because the state has to be invariant under a $U(1)$ symmetry. % $|Q \rangle \rightarrow |Q + 1 \rangle$.
The partition function or respectively the path integral after a wick rotation can then be written as
\begin{align*}
    Z = \sum_{Q = - \infty}^\infty e^{i \theta Q}Z_Q
\end{align*}
where the theta term is evaluated to the factor $e^{i Q \theta}$  and
$Z_Q$ is the partion function for a fixed topological charge $Q$ \cite[Sec. II, Eq. 2.7]{Leutwyler:1992yt}.

\subsubsection{The Zero Temperature Axion Mass}
\label{sec:axion_mass}
In this section the computation of the zero temperature axion mass is presented.
The square of th mass can be calculated as the prefactor of the quadratic term in the series expansion of the axion potential eg. the vacuum expectation value at $\theta = 0$, because this term is the mass term in the lagrangian.
In order to calculate the vacuum energy the partition functions for each topological sector with a fixed $Q$ has to be evaluated.
It can be written as
\begin{align*}
    Z_Q &= \int [\diff G] [\diff q] [\diff \overline{q}] \exp \left( - \underbrace{\int \diff^4 x \mathcal{L}_e}_{= S_E} \right)
     = \int [\diff G] e^{S_G} \det (i \slashed{D} + \tilde{\mathcal{M}})
\end{align*}
where the euclidean Lagrange density $\mathcal{L}_e$ is the wick rotation of $\mathcal{L}_\mathrm{QCD} - \mathcal{L}_\theta$
\cite[Sec. II, Eq. 2.8, 2.9]{Leutwyler:1992yt} and $\tilde{\mathcal{M}} = P_L \mathcal{M} + P_R \mathcal{M}^\dagger $ using the left and right chiral
projection operators $P_L, P_R$.
The determinant of the Dirac operator is computed in \cite[Sec. II, Eq. 2.9]{Leutwyler:1992yt} and given as
\begin{align*}
    \det (i \slashed{D} + \mathcal{M}) = \left( \mathrm{det}_f \mathcal{M} \right)^Q \prod_n \mathrm{det}_f \left( \lambda_n^2 + \mathcal{M} \mathcal{M}^\dagger \right)
\end{align*}
where $\lambda_n$ are the eigenvalues of $\slashed{D}$ who are also computed in \cite{Leutwyler:1992yt}
and the product omits zero modes.
It can now be seen that the $\theta$ dependence of the partition function
is like $\sim \det_f \mathcal{M} e^{i \theta / N_f}$
where $\det_f$ is the determinant over the $N_f \times N_f$ quark mass matrices
in contrast to the functional determinant $\det$ of operators on the fields.
Using this result one can find an effective field theory independent of the
topological charge $Q$ as \cite[Sec. VIII, Eq 8.4]{Leutwyler:1992yt}
\begin{align}
    \label{eq:L_eff}
    \mathcal{L}_\mathrm{eff} = \frac{f_\pi^2}{4} \trace ( \partial_\mu U^\dagger \partial_\mu U) + \Sigma \mathrm{Re} e^{-i \theta / N_f} \trace \mathcal{M} U^\dagger + \, ...
\end{align}
where $U$ is the matrix describing the $N_f^2 - 1$ Goldstone bosons in the vacuum eg. the Pions.
Evaluating this gives \cite[Sec. VIII, Eq. 8.15]{Leutwyler:1992yt}
\begin{align*}
   Z  =  \frac{2}{V \Sigma m_0} \mathrm{I}_1 (V \Sigma m_0)
   \approx \frac{2}{V \Sigma m_0} \frac{1}{\sqrt{2 \pi V \Sigma m_0}} \exp (V \Sigma m_0)
\end{align*}
where $\Sigma$ is an unknown prefactor, $I_1$ is the first purely imaginary Bessel function and $m_0 = |m_u + e^{- i \theta} m_d|$,
where $m_u$ and $m_d$ are the masses of the up and the down Quark.
The expectation value of the energy of the $\theta$ vacuum can now be computed as \cite[Sec. IX, Eq. 9.5]{Leutwyler:1992yt}
\begin{align*}
   \epsilon_0(\theta) = - \frac{1}{V} \log \frac{Z(\theta)}{Z(0)} =
   \Sigma (m_u + m_d) \left( 1 - \frac{\sqrt{m_u^2 + m_d^2 + 2 m_u m_d \cos \theta}}{m_u + m_d}\right)
\end{align*}
From this the axion mass at $T = 0$ can be computed as \cite[Sec. IV, Eq. 9.7]{Leutwyler:1992yt}
\begin{align}
    \label{eq:axion_mass_at_zero_T}
    m_a^2 f_a^2 =
    \left. \frac{\diff^2 \epsilon_0}{\diff \theta^2} \right|_{\theta = 0} = \Sigma \frac{m_u m_d}{m_u + m_d}
    = m_\pi^2 f_\pi^2 \frac{m_u m_d}{(m_u + m_d)^2}
\end{align}
where the unknown factor $\Sigma$ was replaced by computing the pion mass from the partition function of the lagrangian in equation \ref{eq:L_eff} as a function of $\Sigma$,
$f_a$ is axion decay constant and $f_\pi$ is the pion decay constant.
Now since the axion mass is known at $T = 0$ one can use this
for the potential at low $T$ either with the harmonic potential
or using the cosine potential, that will come up in the next
section.

\subsubsection{The High Temperature Potential using the Dilute Instanton Gas Approximation}
\label{sec:DIGA}
In this section the axion potential is computed at high temperatures.
To compute the axion potential at finite temperature one has to
compute the free energy density instead of the energy density \cite[Eq. 2.10]{FiniteTempQCD}.
This is because now the quark-gluon condensate also has thermal energy but not all of it is accessible as potential energy for the axion, only the free energy part is.
\begin{align*}
    F(\theta, T) = - \frac{1}{V} \log \frac{Z(\theta, T)}{Z(0, T)}
\end{align*}
Here the time direction of the four volume $V$ is identified with the inverse temperature by wick rotation.
Now one computes the partition function $Z_1$ for the topological charge $Q = 1$ by using a stationary phase approximation
analogous to
\begin{align*}
    Z_Q = \int [\diff G] [\diff q] [\diff \overline{d}] \exp S_\mathrm{E}(Q) \approx
    \left(  \frac{2 \pi}{S(G_0, q_0, \overline{q}_0)''}  \right)^{1/2} \exp \left( - S(G_0, q_0, \overline{q}_0) \right).
\end{align*}
where $S_0' = 0, S_0 > 0$,
for the fields $G_0, q_0, \overline{q}_0$ who minimize the euclidean action $S_E$. This solution is called the instanton solution.
Instantons are pseudoparticles within the vacuum state.
It can be computed as \cite[Sec. VI]{FiniteTempQCD}
\begin{align*}
    Z_1 = -2 \int \diff^4 \, \rho \, n(\rho)
\end{align*}
where $n(\rho)$ is the instanton size distribution
which is to be computed \footnote{A heroic calculation or a Herculean calculation \cite[Page 346]{Instantons}}
\cite{thooft_PhysRevD.14.3432}
\begin{align*}
    n(\rho) = \mathrm{Some\,very\,ugly\,term}
\end{align*}

Finally the dilute gas approximation is applied to the instanton solution. One writes the topological charge as $Q = n - \overline{n}$ for the number of instantons
$n$ and anti-instantons $\overline{n}$ and approximates the partionfunction as
\cite[Chap. 7, Eq. 3.55]{Symmetry}
\begin{align*}
    Z \approx \sum_{n,\overline{n} = 0}^\infty \frac{1}{n! \overline{n}!} \left(Z_1 e^{i \theta}\right)^n \left(Z_1 e^{- i \theta}\right)^{\overline{n}}
    = \exp \left( Z_1 e^{i \theta} \right) \exp \left( Z_1 e^{- i \theta} \right) = \exp (2 Z_1 \cos \theta)
\end{align*}
where each topological
sector is seen as $n$ instantons and
$\overline{n}$ anti-instantons
who do not interact with each other.
The factorial term comes from
the indistinguishably of the instantons.
Then the free energy density eg. the axion potential at finite temperature is given in this approximation as
\begin{align}
    \label{eq:diga_axion_potential}
    % V(\theta, T) = \frac{2 Z_1}{V} (1 - \cos \theta) = 2 \int_0^\infty \diff \rho D(\rho) G(\pi \rho T) (1 - \cos \theta)
    F(\theta, T) = \frac{2 Z_1}{V} (1 - \cos \theta) = 2 \int_0^\infty \diff \rho n(\rho) (1 - \cos \theta) = \frac{\chi_\mathrm{top}}{f_a^2} (1 - \cos \theta)
\end{align}
where
\begin{align}
    \label{eq:temp_dep_mass_diga}
    Z_1 \sim T^{-(7 + N_f / 3)}
\end{align}
and $\chi_\mathrm{top}$ is called the topological susceptibility with the mass definied as $m_a^2 f_a^2 = \chi_\mathrm{top}$
analogous to the zero temperature mass computation in equation \ref{eq:axion_mass_at_zero_T}
Note that the computed potential diverges at low $T$ and therefore this approximation is not applicable at low temperatures. One uses this
together with the $T = 0$ result and cut the high temperature potential
off once it exceeds the $T = 0$ mass.

\subsection{Comparison of different Axion Mass Approximations and the Lattice Result}
The axion mass at absolute zero can be computed accordingly to section \ref{sec:axion_mass} using chiral pertubation theory \cite[Sec III, Page 7, Eq. 15]{AxionCosmoRev} as
\begin{align}
    \label{eq:m_a_T0}
    m_a(T = 0) = m_\pi f_\pi / f_a \frac{\sqrt{m_u m_d}}{m_u + m_d} m_a \approx 6 \cdot 10^{-10} \cdot \mathrm{eV} \frac{10^{16} \cdot \mathrm{GeV}}{f_a / N}.
\end{align}
where $m_u, m_d, m_\pi$ are the masses of the up and down quark as well as the neutral pion and
$f_\pi$ is the pion decay constant.
For high temperatures one can use the dilute instanton gas approximation, accordingly to section \ref{sec:DIGA}, giving
the mass as a function of temperature parametrized in \cite[Sec. 2.1]{Fox:2004kb} as
\begin{align}
    \label{eq:axion_mass1}
    m_a(T) = m_a C \left( \frac{\Lambda_\mathrm{QCD}}{T} \right)^{n}
\end{align}
for $T > \Lambda_\mathrm{QCD}$.
The exponent $n$ can be computed from equation \ref{eq:temp_dep_mass_diga} and is $n \approx 4$ \cite[Appendix B]{Fox:2004kb}.
There also give a correction factor of
\begin{align}
    \left( 1 - \log \frac{T}{\Lambda_\mathrm{QCD}}\right)^d
\end{align}
with $d \approx 1.2$ to the power law.
Note that this power law behavior is a quite general description, where the parameters, namely the
exponent $n$ and the prefactor $C$ as well as the energy scale involved $\Lambda_\mathrm{QCD}$
can be fitted to more complicated models. \footnote{This might also include non QCD couplings, like to
the hidden sector.}

This is done in the result from Shellard et al. \cite{LatticQCD4Cosmo} to the interacting instanton liquid model.
% All following numerical results are computed using this value.
This alternative approximation using the interacting instanton liquid model, where interactions between instantons in the QCD vacuum are not neglected, results n
 a fit for the low temperature regime as \cite[Sec. III, Page 8, Eq. 19]{AxionCosmoRev}
\begin{align}
    \label{eq:m_a_low_temp_shellard}
    m_a(T)^2 f_a^2 = 1.46 \cdot 10^{-3} \Lambda^4 \frac{1 + 0.5 \cdot T / \Lambda}{1 + (3.53 \cdot T / \Lambda)^{7.48}} \, \, \mathrm{for} \, \, 0 < T / \Lambda < 1.125
\end{align}
as well as \cite[Sec III, Page 9, Eq. 22]{AxionCosmoRev} for the high temperatures
\begin{align}
    \label{eq:m_a_high_temp_shellard}
    m_a(T)^2 f_a^2 = \frac{\alpha_a \Lambda^4}{(T / \Lambda)^n} \, \, \mathrm{for} \, \, T > \Lambda_\mathrm{QCD}
\end{align}
with $\alpha_a = 1.68 \cdot 10^{-7}$ and $n = 6.68$.
The full result for the high temperature regime of this approximation is given as \cite[Sec. III, Eq. 20]{AxionCosmoRev}
\begin{align}
m^2_a f^2_a = \Lambda^4
\left\{
 \begin{array}{l@{,\;}l}
 \exp\left[ d^{(3)}_0 + d^{(3)}_1 \ln\frac{T}{\Lambda} + d^{(3)}_2 \left(\ln\frac{T}{\Lambda}\right)^2 + d^{(3)}_3 \left(\ln\frac{T}{\Lambda}\right)^3 \right] & T^{(3)} < T < T^{(4)}\\
 \exp\left[ d^{(4)}_0 + d^{(4)}_1 \ln\frac{T}{\Lambda} + d^{(4)}_2 \left(\ln\frac{T}{\Lambda}\right)^2 \right] & T^{(4)} < T < T^{(5)}\\
\exp\left[ d^{(5)}_0 + d^{(5)}_1 \ln\frac{T}{\Lambda} + d^{(5)}_2 \left(\ln\frac{T}{\Lambda}\right)^2 \right] & T^{(5)} < T < T^{(6)}\\
\end{array}
\right. ,
\end{align}
where the constants are given
in \cite[Sec. III, Eq. 21]{AxionCosmoRev}.    \footnote{There is a typo in the parameters in \cite[Sec. III, Eq. 21]{AxionCosmoRev}, $d_0^{(4)}$ should be negative but is given as positive. The positive version produces drastically wrong results.}
In those formulas from Shellard et al. in \cite{LatticQCD4Cosmo} $\Lambda$ is taken as $\Lambda = 400 \, \mathrm{MeV}$.

The full dependence of the mass of the axion on the temperature is computed in \cite{LatticQCD4Cosmo}
using lattice QCD methods as the topological susceptibility $\chi_\mathrm{top}(T) = m_a(T)^2 f_a^2$ as also seen in equation \ref{eq:diga_axion_potential}
and is given as a table of numerical values.
A graphical comparison of all of those result is shown in figure \ref{fig:m_a_of_T_plot}
including the relative error with respect to the lattice result.
All particle parameters are taken from \cite{PDG}.
\begin{figure}
    \centering
    \includegraphics[width=\linewidth]{m_of_T_plot.pdf}
    \caption{Comparison of different approximation and the lattice result for the axion mass as a
    function of the temperature for $f_a = 10^{12} \, \mathrm{GeV}$.
    The legend on the left holds for both plots. Where on the left the axion mass as a function of
    the temperature is shown, while on the right the relative error in respect to the lattice result
    is shown.}
    \label{fig:m_a_of_T_plot}
\end{figure}

\subsection{Approximation for the Relic Density of the QCD axion}
\label{sec:wkb_qcd}
Now the axion relic density can be analytically computed for the QCD axion like
in section \ref{sec:wkb_alp} for a general axion like particle. Raditation dominatation
is assumed.
First, one needs to find the temperature of the universe at the point where the
axion field starts to oscillate.
Using the oscillation condition from formula \ref{eq:a_osc_cond},
the energy density for radiation from equation \ref{eq:energy_denstiy} and the Friedmann Equation in formula \ref{eq:friedmann_equation}
one obtains the relation
\begin{align*}
    T_\mathrm{osc} = \left( \frac{10 m_a(T_\mathrm{osc})^2 M_\mathrm{pl}^2}{\pi^2 g_{*, \rho}(T_\mathrm{osc})} \right)^{1/4}
\end{align*}
Substituting the axion mass with the expression from equation \ref{eq:axion_mass1} yields
\begin{align}
    \label{eq:T_osc1}
    \notag & T_\mathrm{osc} = \left( \frac{10 M_\mathrm{pl}^2}{\pi^2 g_{*, \rho}(T_\mathrm{osc})} \right)^{1/4} \left( m_a C \left( \frac{\Lambda_\mathrm{QCD}}{T_\mathrm{osc}} \right)^{n} \right)^{2 \cdot \frac{1}{4}}
    \\ & \Rightarrow T_\mathrm{osc} = \underbrace{\left( \left( \frac{10 M_\mathrm{pl}^2}{\pi^2 g_*} \right)^{1/4} C^{1/2} \Lambda_\mathrm{QCD}^{n/2} (6 \cdot 10^{-10} \, \mathrm{eV})^{1/2} \right)^{1/(1 + n/2)} }_{=: \kappa \approx 150 \, \mathrm{MeV} } \left( \frac{10^{16} \, \mathrm{GeV}}{f_a / N} \right)^{1/2 \cdot 1/(1 + n/2)}
\end{align}

Now the number density of the axions is computed using equation \ref{eq:axion_energy_density_and_pressure} for the axion energy density as
\begin{align*}
    n_a(T_\mathrm{osc}) = \frac{\rho_a}{m_a} = \frac{f_c}{2} m_a(T_\mathrm{osc}) \langle \theta_i^2 \rangle \left( \frac{f_a}{N} \right)^2
\end{align*}
where $f_c$ is a correction factor for the temperature dependent mass for the parameterization from equation \ref{eq:axion_mass1} following \cite[Sec. 2.1]{Fox:2004kb}.
Now using equation \ref{eq:n_over_s}
\begin{align*}
    n_a(\mathrm{today}) = \gamma \frac{s(T_0)}{s(T_\mathrm{osc})} n_a(T_\mathrm{osc}),
\end{align*}
where $\gamma$ is a correction factor for entropy release during the expansion of the universe.
The entropy density is computed using equation \ref{eq:entropy_density}.
Therefore the axion relic density parameter is given as
\begin{align}
    \label{eq:relic_density_high_temp}
    \Omega_a h^2 = \underbrace{
    \frac{h^2 C (6 \cdot 10^{-10} \, \mathrm{eV})^2 (10^{16} \, \mathrm{GeV})^2 T_0^3 f_c g_{*, s}(\mathrm{today}) \Lambda_\mathrm{QCD}^n}{2 g_{*, s}(T_\mathrm{osc}) \rho_c \kappa^{n + 3}}
    }_{\approx 2 \cdot 10^4} \cdot \left( \frac{f_a / N}{10^{16} \, \mathrm{GeV}}\right) ^ {\frac{n + 3}{n + 2}} \langle \theta_i^2 \rangle \gamma
\end{align}
A plot for the relic density can be found in figure \ref{fig:qcd_wkb_plot}.
\begin{figure}
    \centering
    \includegraphics[width=0.7\linewidth]{qcd_relic_denstiy_wkb_plot.pdf}
    \caption{Axion relic density parameter for the QCD axion with $T_\mathrm{osc} > \Lambda_\mathrm{QCD}$
    as a function of the initial field value $\theta_i$ and the axion decay constant $f_a$.}
    \label{fig:qcd_wkb_plot}
\end{figure}
The same computation can done for $T_\mathrm{osc} < \Lambda_\mathrm{QCD}$ by assuming that $m_a(T) \approx m_a(T = 0)$ \cite[Sec. 2.1]{Fox:2004kb}
\begin{align*}
    T_\mathrm{osc} \sim 950 \mathrm{MeV} \, \left(\frac{10^{16} \, \mathrm{GeV}}{f_a} \right)^2 \, \left(\frac{m_a(T_\mathrm{osc})}{m_a} \right)^{1/2}
\end{align*}
resulting in
\begin{align}
    \label{eq:axion_mass2}
    \Omega_a h^2 \simeq 5 \cdot 10^3 \left( \frac{f_a / N}{10^{16} \, \mathrm{GeV}} \right)^{3/2} \langle \theta_i^2 \rangle \gamma f_c \left( \frac{m_a(T_osc)}{m_a} \right)^{-1}.
\end{align}

\section{Numerical Simulation of the QCD Axion Relic Density}
In this section, the method for the numerical solution of the equation of motion and the
computation the axion relic density is discussed.
Without the harmonic approximation, one can not solve the equation of motion
analytically, so one has to rely on numerical solutions.

\subsection{Temperature Time Dependence}
In order to solve the equation of motion numerically
on needs the time dependence of the temperature
to evaluate the axion mass at a given time.
Alternatively the full equation of motion can be rewritten
to have the temperature as the independent variable.
This approach is taken from \cite[S11]{LatticQCD4Cosmo}.
In both situations the derivative $\frac{\diff t}{\diff T}$
is needed.

\noindent
One uses the Gibbs Duhem relation for intrinsic quantities
\begin{align}
    \label{eq:gibbs_duhem}
    \rho + P = T s + \mu n,
\end{align}
where $P$ is the pressure, $\rho$ the energy density, $T$ the temperature, $s$ the entropy density,
$n$ the number density and $\mu$ the chemical potential.
In combination with the temperature dependence of the entropy density in equation
\ref{eq:entropy_density}
and the friedmann equation
in equation \ref{eq:friedmann_equation} as well as the continuity equation \ref{eq:cont}
together with the equation for
the energy density in formula \ref{eq:energy_denstiy}
to compute
% \begin{align*}
%     \frac{\diff ^2 \phi}{\diff  t^2} + 3 H \frac{\diff  \phi}{\diff  t} + V'(\phi) = 0
% \end{align*}
% \begin{align*}
%     \frac{\diff  t}{\diff  T}
%     &= \frac{\diff  \rho}{\diff  T} \frac{\diff  t}{\diff  \rho}
%     = \frac{\frac{\diff  \rho}{\diff  T}} {\frac{\diff  \rho}{\diff  t}}
%     = \frac{\frac{\diff }{\diff T} \frac{\pi^2}{30}g_*(T) T^4}{-3HTs}
%     = \frac{\pi^2}{30} \frac{g_*'(T)T^4 + 4 g_*(T) T^3 }{-3HTs} \\
%     &= \frac{\pi^2}{30} \frac{g_*'(T)T^4 + 4 g_*(T) T^3 }{-3T \frac{2 \pi^2}{45} g_{s, *}(T) T^3 \sqrt{\frac{8 \pi \rho}{3 M_\mathrm{pl}^2}}} \\
%     &= \frac{\pi^2}{30} \frac{g_*'(T)T^4 + 4 g_*(T) T^3 }{-3T \frac{2 \pi^2}{45} g_{s, *}(T) T^3 \sqrt{\frac{8 \pi}{3 M_\mathrm{pl}^2} \frac{\pi^2}{30} g_{\rho, *} T^4}}
% \end{align*}
\begin{align}
    \label{eq:dtdT}
    \frac{\diff t}{\diff T} = - M_\mathrm{pl} \sqrt{\frac{45}{64 \pi^3}} \frac{1}{T^3 g_s(T) \sqrt{g_\rho(T)}} (T g_\rho'(T) + 4 g_\rho(T)).
\end{align}
Note that this formula is only valid in the radiation dominated universe.
For other scenarios one has to find different relations for the temperature evolution.

\subsection{The Relativistic Degrees of Freedom}
In order to solve equation \ref{eq:dtdT} we need the quantity $g_s(T)$ as well as $g_\rho(T)$. They can be found as a numerical table like in \cite[Table S2]{LatticQCD4Cosmo}.
For the numerical integration of equation \ref{eq:dtdT} needs them as continuous and differentiable
functions. This can be achieved by interpolating them using a cubic spline interpolation, but because this would create
some "bumps" in the interpolation function a Pchip interpolator from the Python library \textit{SciPy} is used. This interpolator uses cubic splines as well but loses
the continueness of the second derivative in favour of only using monotonic cubic functions for the interpolation, avoiding those artifacts in the interpolation.
A plot of the data from \cite[S4.3, Table S2]{LatticQCD4Cosmo} can be found in figure \ref{fig:g_plot}
together with the used interpolation.
Because for the later steps in the course of the
bachelor project, also the energy range at $~ 1 \, \mathrm{MeV}$ is needed.


\subsection{The Temperature Evolution in the Radiation Dominated Universe}
Now equation \ref{eq:dtdT} can be integrated numerically
starting at some initial time $t_0$ and from
$T_0$ to $T_\mathrm{end}$ resulting in the time $t$ as a function of temperature $T$
\begin{align*}
    t(T) = t_0 + \int_{T_0}^{T_\mathrm{end}} \frac{\diff t}{\diff T} \diff T.
\end{align*}
The numerical integration is done using the \textit{odeint} routine
from the python library \textit{SciPy}. The result is shown
as a log log plot in figure \ref{fig:T_of_t_plot}.
Currently $t_0$ is set to $t_0 = 0$.
\begin{figure}[H]
    \centering
    \begin{tabular}{cc}
        \subfloat[\label{fig:g_plot}]{\includegraphics[width=0.5\linewidth]{g_plot.pdf}} &
        \subfloat[\label{fig:T_of_t_plot}]{\includegraphics[width=0.5\linewidth]{T_of_t_plot.pdf}}
    \end{tabular}
    \caption{In the left figure the interpolated values for the effective number of relativistic degrees of freedom for the energy and the entropy density are plotted against temperature. On the right the numerically integrated
    time temperature dependence from equation \ref{eq:dtdT} is shown.}
\end{figure}

\subsection{The Equation of Motion as a Function of Temperature}
As an alternative to solving equation \ref{eq:eom} and
equation \ref{eq:dtdT} simultaneously, one can
solve the equation of motion as a function of $T$ itself using equation \ref{eq:dtdT},
that is to change in equation \ref{eq:eom} all derivatives to derivatives in $T$.
The first derivative is
\begin{align*}
    \frac{\diff  \theta}{\diff  t} = \frac{\diff  \theta}{\diff  T} \frac{\diff  T}{\diff  t}
\end{align*}
and the second
\begin{align*}
    \frac{\diff ^2 \theta}{\diff  t^2} &= \frac{\diff }{\diff  t} \frac{\diff  \theta}{\diff  t}
    = \frac{\diff }{\diff  t} \frac{\diff  T}{\diff  t} \frac{\diff  \theta}{\diff  T}
   % = \frac{\diff ^2 T}{\diff  t^2} \frac{\diff  \theta}{\diff  T} +
    %  \frac{\diff  T}{\diff  t} \frac{\diff }{\diff  t} \frac{\diff  \theta}{\diff  T}
    %= \frac{\diff ^2 T}{\diff  t^2} \frac{\diff  \theta}{\diff  T} +
      %\frac{\diff  T}{\diff  t} \frac{\diff }{\diff  T} \frac{\diff  \theta}{\diff  t} \\
    %&= \frac{\diff ^2 T}{\diff  t^2} \frac{\diff  \theta}{\diff  T} +
    %  \frac{\diff  T}{\diff  t} \frac{\diff }{\diff  T} \frac{\diff  T}{\diff  t} \frac{\diff  \theta}{\diff  T}
    %= \frac{\diff ^2 T}{\diff  t^2} \frac{\diff  \theta}{\diff  T} +
    %  \left( \frac{\diff  T}{\diff  t} \right)^2 \frac{\diff^2  \theta}{\diff  T^2} \\
    %&= \left( \frac{\diff}{\diff t} \frac{\diff T}{\diff t} \right) \frac{\diff  \theta}{\diff  T} +
    %  \left( \frac{\diff  t}{\diff  T} \right)^{-2} \frac{\diff^2  \theta}{\diff  T^2}
  %  = \left( \frac{\diff}{\diff t} \left( \frac{\diff t}{\diff T} \right)^{-1} \right) \frac{\diff  \theta}{\diff  T} +
   %   \left( \frac{\diff  t}{\diff  T} \right)^{-2} \frac{\diff^2  \theta}{\diff  T^2} \\
  %  &= - \left(\left( \frac{\mathrm{d}}{\mathrm{d} t} \frac{\diff t}{\diff T} \right)  \left( \frac{\diff t}{\diff T} \right)^{-2} \right) \frac{\diff  \theta}{\diff  T} +
   %   \left( \frac{\diff  t}{\diff  T} \right)^{-2} \frac{\diff^2  \theta}{\diff  T^2} \\
    = - \left(\frac{\diff T}{\diff t} \frac{\diff^2 t}{\diff T^2} \left( \frac{\diff t}{\diff T} \right)^{-2} \right) \frac{\diff  \theta}{\diff  T} +
      \left( \frac{\diff  t}{\diff  T} \right)^{-2} \frac{\diff^2  \theta}{\diff  T^2}.
\end{align*}
Substituting the derivatives in equation \ref{eq:eom} then yields  \cite[Section S11]{LatticQCD4Cosmo}
\begin{align}
    \label{eq:eom_T}
    % & \frac{\diff ^2 \theta}{\diff  t^2} + 3 H \frac{\diff  \theta}{\diff  t} + V'(\theta) = 0 \\
    % \notag & \left(\frac{\diff T}{\diff t} \frac{\diff^2 t}{\diff T^2} \left( \frac{\diff t}{\diff T} \right)^{-2} \right) \frac{\diff \theta}{\diff T} + \left( \frac{\diff t}{\diff T} \right)^{-2} \frac{\diff^2 \theta}{\diff T^2} + 3H\frac{\diff T}{\diff t}\frac{\diff \theta}{\diff T} + V'(\theta) = 0
    \frac{\diff^2 \theta}{\diff T^2} +
                 \left(
                 3H \frac{\diff t}{\diff T}
                 - \frac{\diff^2 t}{\diff T^2} / \frac{\diff t}{\diff T}
                 \right) \frac{\diff \theta}{\diff T} +
                 V'(\theta) \left( \frac{\diff t}{\diff T} \right)^2 = 0.
\end{align}
In the rest of the project this equation will solved numerically to $T_\mathrm{osc}$ and
further integrated for some oscillations.
Then the quantity $n/s$ will be numerically taken as a mean over several oscillations
and then used to compute the axion relic density in the present. This allows to explore more
complicated potentials and cosmologies (eg. the evolution of $H$ and $\diff t / \diff T$) that can not be solved analytically.




\subsection{Integration of the Equation of Motion}
In general one can not solve the equation of motion analytically.
For instance in order to account for anharmonic effects due to
the cosine potential of the QCD axion, the equation of mtion
has to be integrated numerically. In this section
the implementation of the numerical algorithm to
compute the axion relic density is described.

\noindent
The integration of the equation, namely in the temperature
form \ref{eq:eom_T} is done with
the ????????????? algorithm ????????? using the
ODE solver from the python library SciPy ???????.
The integration is started from $5 T_\mathrm{osc}$ as suggested
in ?????????????? and carried out until the first zero
crossing of the field.
The value of $T_\mathrm{osc}$ is computed numerically from eq. ??????? using a value of $N = 3$.
Note that this is only used to determine the starting point $T_\mathrm{initial}$ for the integration.
The beginning of the oscillation is established from the zero crossing.

\noindent
After that the field starts to oscillate.
Then after a few oscillations the ratio $n / s$ should be conserved as discussed in section ????????.
To make sure that is is the case, the ratio $n / s$ is computed successively over intervals of length ???????
until the ratio $n / s$ is constant, ie. $\frac{\diff n / s}{\diff T} < \epsilon = ??????$ and
there are at least $N_\mathrm{osc} = ??????$ oscillations $= 2 N_\mathrm{osc} + 1$ zero crossings in the interval.

\noindent
During the integration is is required that $T > T_\mathrm{eq}$, since below the temperature at matter radiation equality $T_\mathrm{eq}$
the assumption of radiation domination for the Friedmann equation and the time temperature dependence is not valid
and would lead to wrong results. Therefore this condition has to be checked in order to obtain reliable results.

\noindent
From the value of $n / s$, the relic density of the axion is then computed by scaling to the value of $s$ today as discussed in section ????.
The value of the entropy density $s$ today is dominated by photons and neutrinos ???????. Therefore the entropy density today is given as
\begin{align}
The temperature of the neutrinos can be computed, after $e^+ e^-$ annihilation and neutrino decoupling ie. also today, as
\begin{align}
    T_\nu(\mathrm{today}) = ??????.
\end{align}

\noindent
A sample field evolution, computed numerically, can be found in figure \ref{fig:sample_field_evo}.
\begin{figure}[H]
    \centering
    \includegraphics{}
    \caption{Caption}
    \label{fig:sample_field_evo}
\end{figure}

\noindent
For further analysis, the relic density is computed for a range of initial field values $\theta_i$ and axion decay constants $f_a$.
The resulting density is plotted in figure \ref{fig:numerical_QCD_result}. The white region marks the area where the relic density is higher that the
DM density.
\begin{figure}[H]
    \centering
    \includegraphics[width=\linewidth]{numerical_relic_density.pdf}
    \caption{Caption}
    \label{fig:numerical_QCD_result}
\end{figure}

\section{Analysis of the Simulated Result}
In this section the results obtained from the numerical simulation of the QCD axion density are analyzed.

\subsection{Comparison with the Analytic Result}
The relative error to the analytic approximation
as computed in section ??????? shown in figure \ref{fig:analytic_numerical_comparison}.
The numerical result was computed using the harmonic
potential, since the goal is to compare the approximation and not the derivation from
the anharmonic case.
\begin{figure}[H]
    \centering
    \includegraphics{}
    \caption{Caption}
    \label{fig:analytic_numerical_comparison}
\end{figure}

\subsection{Anharmonic Corrections}
The influence of the anharmonic correction is made quantative
by the anharmonic correction function $F_\mathrm{anhamr.}(\theta_i)$.
By computing the relic density using the harmonic potential resulting in $\Omega_\mathrm{a, harmonic}$ as well as the cosine potential resulting in $\Omega_\mathrm{a, cosine}$, the
anharmonic correction $F(\theta)$ defined as
\begin{align}
    F(\theta) = \frac{\Omega_\mathrm{a, cosine}}{\Omega_\mathrm{a, harmonic}},
\end{align}
can be computed.
The anharmonic corection is $F(\theta) = 1$ for $\theta_i = 0$ and
monotonically increases with $\theta_i$ and
diverges at $\theta_i = \pi$.
This correction is plotted in figure \ref{fig:anharmonic_correction_plot}.
In this plot a analytical fit for the anharmonic correction,
as given in ????????, is shown. It is given as
\begin{align}
    F_\mathrm{anharm.}(\theta_i) = 1.
\end{align}
\begin{figure}[H]
    \centering
    \includegraphics[width=\linewidth]{anharmonic_corrections_plot.pdf}
    \caption{Caption}
    \label{fig:anharmonic_correction_plot}
\end{figure}

\subsection{QCD Axion as Dark Matter}
In order to use axions to explain all of the DM, the relic density has to be equal to the measured
DM density. In figure \ref{fig:axion_as_DM_plot} a plot of the initial axion field value $\theta_i$ that produces
the right amount of dark matter as a function of the axion decay constant $f_a$ is shown.
This can be used to validate the simulation, by comparing the resulting plot with the plot given
in ????????. Those results seem to agree.
% ?????????????????? The initial angle of the axion field has to be
\begin{figure}[H]
    \centering
    \includegraphics[width=\linewidth]{initial_theta_for_DM.pdf}
    \caption{Caption}
    \label{fig:axion_as_DM_plot}
\end{figure}


\subsection{Estimation of Particle Physics Paramters}


\section{Hidden Sector Couplings}

\section{Axions from String Decay}

\section{Conclusion}

\appendix

\section{Derivation of the Equation of Motion}

\section{Equation of Motion as a Function of Temperature}

\section{MCMC Plots}



\newpage
%\bibliography{Shared.bib}{}
%\bibliographystyle{plain}
\printbibliography

\end{document}
